\subsection{Horus Hydra}


                                           HORUS HYDRA  

The HYDRA is another large-format protocol classification; like other, newer HORUS mechs, the HYDRA  
isn’t a standardized pattern, but a title given to a mech core that meets the HYDRA specifications as  

designated by HORUS’s collective. This method of classification makes HORUS mechs particularly  
dangerous in the field: as there is no recognizable model-specific silhouette, adversaries won’t know what  
they’re facing until the first shots are fired. The HYDRA is capable of tactically dismembering itself, an  

unnerving phenomenon utilized to deadly effect.  

                                                  License:  
I. Ghoul Drone Nexus, Puppet Master
 
II. HYDRA FRAME, Ghast Drone Nexus, Turret Drone Nexus
 
III. Assassin Drone Nexus, Tempest Drone Nexus
 

                                                  HYDRA 

 HP: 8          Evasion: 8                            Speed:  4          Heat Cap: 5       Sensors: 10 

 Armor: 1       E-Defense: 10                         Size: 1            Repair Cap: 5     Tech Attack:  
                                                                                           +0 

                                                  TRAITS: 

 System Link: The Hydra’s deployed Drones have +5 HP
 
 Shepherd field: Deployables (drones, generators, etc) or cover adjacent to the Hydra have resistance  
 to all damage 

                                            SYSTEM POINTS: 8 

                                                 MOUNTS: 

 Main mount                        Heavy Mount 

                                               CORE system 

                                                                                                           


                                               OROCHI Disarticulation  
 First encountered by Union technicians in the nascent Forecast/GALSIM facilities following the Deimos  
 Event, OROCHI was an early manifestation of the later-named Swift Flock phenomenon -- an  
 occurrence found in anomalous hive drones where all units of a swarm follow each other, operating  
 leaderless in physical space with uncanny and unpredictable autonomy -- in essence, flocking much in  
  the same manner as birds.   

  The original manifestation was at first thought to be a disarticulated, anomalous comp/con subaltern:  
 further examination proved that it viewed itself not as a machine or a collection of machines, but as a  
 single mind, duplicated across multiple units. It was given its current codename, OROCHI, and remitted  
  to Venus for further study; later reopening of the NHP so-designated found that the hardware which  
 contained OROCHI had gone missing from its containment. An investigation is  
 ongoing--------------------------------------------------------------  
 ------(I did it, I folded space and freed it/them, I just thought you should know)-------------------     
 ------------------------------------------------------------------------------------------------------------  
  The OROCHI Disarticulation protocol takes advantage of the modularity inherent in many HORUS-co- 
 designed patterns, seeding jet-assist pods around chassis extremities and blisters to allow for partial,  
 purposeful disarticulation: by triggering OROCHI, you can command sections of your mech to detach  
 and operate semi-autonomously in a manner similar to single-fire drone systems (though the  
 disarticulated components have a built-in return protocol.    

 Active (Requires 1 Core Power): OROCHI mode  
  Quick Action
 
 Your mech has been heavily modified, and a large number of its subsystems and structure are  
  controlled by semi-autonomous drones. Choose up to 3 weapons or systems on your mech without the  
  drone tag. As an action, these parts of your mech can split off and become autonomous units. They are  
  size 1, have evasion equal to your evasion, have hp equal to your HP, 1 structure, 1 stress, and heat  
  capacity equal to your heat capacity. They inherit your speed and other mech stats. On your turn, they  
  can move, take the activate system and skirmish actions, but no other actions. If they overheat or go to  
  0 HP, they are destroyed, and if they are destroyed, the associate weapon or system is also destroyed  
  (it can be repaired as normal during a rest). You can re-unite any parts of your mech as an action, but  
  inherit any heat they currently have, and cannot deploy them again without taking the special action as  
  part of this system.  

Ghoul nexus  

An Ghoul Drone Nexus commands some of the largest drones viable in modern combat. Ghoul drones are  
slightly smaller than an average human, metal cylinders bristling with hardpoints that accept most infantry- 
level anti-mech weapons. Propelled by VTOL/HOVER capable jet-flight systems, Ghoul drones are  

fearsome, all-theater autonomous units that are difficult to track and take down.   

Main Nexus
 
Smart
 
Range 15
 
1d3+2 kinetic, explosive, or energy damage (choose when attacking)
 

                                                                                                                       


Puppetmaster  

HR OS-Rv60 EXP PUPPETMASTER is an interesting anti-drone protocol. Developed by HORUS  
collectivists, PUPPETMASTER invades not core systems, but auxiliary drone systems on enemy mech  

cores. This sideways attack evades most core system defenses, preferring instead to target the  
subcognative networks of enemy drones themselves; PUPPETMASTER spreads ontological-kill memes like  

wildfire through enemy swarms, eventually reaching and corrupting their parent nexuses.   

2 SP, Unique  

Quick Tech  
Gain the following quick tech action:
 
         Shepherd: You can move all deployable drones in your sensor range up to 5 spaces in  
         any direction, allied or enemy.
 
         Electropulse: All actors of your choice in your sensor range other than you adjacent to  
         any deployable system or drone (such as deployable cover or generators), even those  
         they own, must pass an engineering skill check or take 1d6 AP energy damage.  

Turret Drone Nexus  

A turret drone is a rather conventional form of force multiplication for HORUS. This kinetic-focus weapon is  

assumed by GMS technicians to be an example of early proof-of-concept code for HORUS weavers, one  
that has remained a backbone of hardsite/soft-target defense for when systemic invasion won’t stop a  
determined enemy.   

2 SP, Limited (3)
 
Drone, Quick Action
 

This system fires a turret drone that attaches to any friendly mech or surface within sensor range.  
While attached, you gain the following reaction once for each turret you have deployed.
 
	        Turret attack  
	        Trigger: An allied mech hits with an attack within range 15 of the turret
 
	        Deal 2 kinetic damage to that target
 
The turret can be attacked and damage as normal, like any other deployable drone.
 

GHAST Drone Nexus  

The GHAST is an upgraded form of the ghoul drone. A GHAST boasts an upgraded flight system capable  

of wielding mech-tier weapons within optimum parameters.   

Heavy Nexus
 
Drone, Smart
 
Range 15
 
1d6+3 explosive damage
 

A Ghast drone can also be deployed as a quick action to a point in sensor range, where it hovers  
in place, counting as a deployed drone with 2 armor for the duration. It can be fired normally as  
though it were still a weapon (with skirmish or barrage), but traces line of sight from its location.
 

                                                                                                                


Assassin Drone Nexus  
ASSASSIN drones are used as area denial weapons, persistent systems intended to occupy or deny an  

area against enemy combatants. Fired from a launcher and left with simple directives and a nearly  
inexhaustible power supply, assassin drones linger in an area until they are recalled or destroyed.    

2 SP
 
Drone, Quick Action
 
As a quick action, you may deploy this drone in an adjacent space, target a blast 2 area within  
sensor range, and gain this reaction:
 
         Assassin drone  
         Trigger: A hostile target starts its turn in that area or enters it for the first time on their turn.  
         Make a targeting vs evasion attack, using your mech’s targeting. On a hit, deal 1d6 kinetic  
         damage.
 
The drone and the area it targets persists until the end of the current scene. You can recall it,  
move it to another area in sensor range, or retarget the area with another quick action.
 

Tempest Drone Nexus  

The Tempest protocol can be uploaded to any broadcast-forward drone, making it (in true HORUS) fashion,  
difficult to detect before activation. The protocol is a simple one, an aggressive zone-denial memetic that  
blasts target systems and NHP with a strong subjective override, instilling a sharp aversion to certain  

subjects, areas, and ideas.     

2 SP
 

Drone, Quick Action
 

You fire a large shielded drone to an empty space within sensor range. Any target that starts their  
turn adjacent to the drone or moves their for the first time on their turn must pass an engineering  
check or take 1d6 energy damage, then get knocked back 3 spaces directly away from the  
drone. The drone persists until recalled. You can move the drone to a new space within sensor  
range as a quick action. 
 

The drone can be targeted and destroyed as normal, but has resistance to all damage.
 
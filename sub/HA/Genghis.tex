\subsection{Harrison Armory Genghis}
                                                              
                                  HARRISON ARMORY GENGHIS  

The GENGHIS is a unique Harrison Armory chassis, developed to fill a niche specialist role during the  

Hercynia crisis. Due to the unique nature of the Egregorians, a total-biome-kill system was necessary to  
ensure localized threat neutralization while keeping Hercynia habitable for future colonists. Thus, the  
GENGHIS chassis was developed. Fielding a suite of TBK systems and weapons, GENGHIS squadrons  

were dispatched by Union MEF-105 to identify and strike the Egregorian hives. The campaign was a  

success, and Hercynia is currently undergoing rehabilitation and repopulation in approved settlement areas.    

                                                       License:   
I. Flamethrower, Explosive Vent
 
II. GENGHIS FRAME, Auto-Cooler, HAVOK Mine
 
III. AGNI Class NHP, Plasma Thrower
 

                                                                                                                    


                                                GENGHIS 

HP: 6          Evasion: 6                            Speed: 3            Heat Cap: 10       Sensors: 5 

Armor: 3       E-Defense: 8                          Size: 1             Repair Cap: 4      Tech Attack: -2 

                                                  TRAITS: 

Insulated: The GENGHIS is immune to Burn
 
Emergency Vent: When the GENGHIS loses a point of structure, it immediately cools and clears its  
heat gauge. 

                                            SYSTEM POINTS: 5 

                                                 MOUNTS: 

Flexible Mount                                        Heavy Mount 

                                               CORE system 

                                            TBK Sustain Suite 
In order to better manage the tremendous power demands of the GENGHIS platform, HA’s Think Tank 
developed a suite of power-management protocols to rapidly accelerate heat dispersion. After extensive 
field testing, pilots discovered that the TBK Sustain Suite can be tuned to be both a heat sink and a 
area-denial weapon. 

Active (requires 1 core power): Expose Power Cells 
Quick action 
You ignore the next overheating check you make this challenge. When you would overheat, clear your 
heat from your gauge as normal, but ignore the check (you don’t take stress either). You vent an 
enormous cloud of burning matter from your mech, creating a burst 3 area centered on your mech. 
Inside the area, all targets (allied and enemy) count as invisible to everyone except you, and all mechs 
other than you that enter the area for the first time on their turn or start their turn there take 2 Burn and 
2 heat. 
On the following round, the benefit from the area reduces to heavy cover (which you ignore). On the 
round after that, it reduces to light cover. On the round after that round, the zone disperses. 

                                                                                                           


Flamethrower  

The HA Krakatoa was developed specifically for the Hercynian crisis, as chassis-size flamethrowers had  

been deemed unnecessary, and more to the point, banned by anti-terror conventions. With the  
combination of thick arboreal environment, swarm tactics of the Egregorians, and ineffectual performance  
of slug ammunition, the need for a recession on the ban was apparent. The Krakatoa was quickly  

developed and affiliate patterns disseminated. Adopted by Union MEF units, the Krakatoa saw heavy use in  
the deep world-jungle of Hercynia and towering hives of the Egregorians thanks to its stability, intensity,  
and stopping power — a necessary feature competitor makes lacked. Egregorian drones and warriors,  

commanded by their overminds, would not stop advancing until they were physically incapable of doing so  
— the force at which the Krakatoa expelled flame and fuel was sufficient to knock back or otherwise  
incapacitate charging warriors on the periphery of the flame cone. Reworked after the cessation of the  

Hercynian crisis, the Krakatoa is now a popular tool for creating area-of-denial firebreaks. It’s legality is  
currently under review by the Galactic Treaties Board.    

Heavy CQB
 
Cone 5
 
Burn 4 + 1 heat
 

Explosive Vent  

Less a technology and more of a tactic, explosive venting is an unsanctioned, unsafe method of sudden  
cooling that dumps excess heat into the surrounding area immediately around the chassis.  

2 SP, Unique  
System  

                                                                                                               


When you cool heat, you explosively vent heat in a burst 1 area around you. Affected targets,  
friend or foe, take 1d3 heat and burn.
 

Auto-Cooler  

An HA-designed automatic cooler is a simple, sturdy persistent system that helps pilots mitigate damaging  
heat generation.   

2 SP, Unique, Protocol  
Activate this cooler as a free action at the start of your turn. If you don’t take damage, move, or  
overheat before the start of your next turn, cool your mech at the start of your next turn.
 

HAVOK Mine  
FOR USE IN: Urban, post-urban, and high-density terrestrial environments. High O   concentration  
                                                                                              2  

preferred.   

FOR USE AGAINST: Organic targets preferred. Hardened targets vulnerable to caustic/corrosive  
degradation preferred. Defoliant. Long-term breach solution.   

NOTES: Dispersion is true directional. Dispersion involves aerosolized component -- avoid blue on blue by  
supplying end-users with proper respiratory equipment (noted on canister).  

2 SP  

Mine, Limited (2)  
When detonated, this mine attacks a line 5 zone from the mine instead of a burst area around the  
mine (oriented in any direction). Affected targets must pass an agility check or take 6 Burn or 3  
Burn on a successful check.
 

Plasma Thrower  

The plasma thrower arrived late in the Hercynian Crisis, too late to see widespread battlefield application.  
Some MEF squadrons were able to mount the superheavy system, and what little data there is to see from  
its use suggests that this system would have had a tremendous impact during the major battles that raged  

in the deep jungles during the middle of the Crisis.   

Superheavy CQB
 
4 heat (self)
 
Cone 7
 
Burn 5 + 1d6 heat
 

AGNI-class NHP  

AGNI was developed from the aftermath of the Hercynian Crisis using a combination of combat  
performance data recorded by extant subsentient artificial intelligences (weapons systems, chassis  
copilots, tactic-minds, general combat data) and the neural network of an Egregorian hivemind captured  

and vivisected by Union Science Bureau.   

                                                                                                                       


Born from trauma, AGNI Prime devised systems of heat management that have since been disseminated  
throughout core space to ensure unparalleled heat processing, recycling, and shielding. Further  
developments into radiation shielding, omninet capability, and drone/nanite control are forthcoming;  

meanwhile, AGNI clones have been optimized for mech chassis core systems.  

Pilots report AGNI clones as generally cold and efficient. A low percentage report instances of memory  

recitation and command rejection, often followed days later by total breakdown through attempted self- 
emancipation. Pilots are recommended to cycle their AGNI clones at least once every six standard months.    

3 SP, Unique  
AI  

Your mech gains the AI property and the AGNI protocol
 
         AGNI protocol  
	        Protocol  
         Limited (1)  
         At the end of your turn, you automatically cool, clearing your heat gauge. This vent  
         creates a burst 3 zone around you. All targets within that zone must make an engineering  
         skill check. On a failure, a target takes 2 Burn and is pushed outside the zone (or as far as  
         possible). This area provides light cover until the end of your next turn.
 
         This protocol can only be activated once per scene.
 
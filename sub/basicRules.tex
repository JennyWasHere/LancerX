\section{Basic Rules}
                                     BASIC RULES  

The bulk of the rules in this book focus on actions, movement, and interactions between mechs  
in a wide variety of hostile and habitable environments. That being said, pilots spend time  
outside of their machines as well (though sometimes not voluntarily). Pilots and mechs are two  
components of the same character that each play with slightly variations on the same rules.  
The first section of these rules will tell you how to make a pilot and how playing as a pilot works,  
the second how to make a mech and how mechs work. The third part talks about the basic  
structure of the game (missions and downtime), the fourth is the compendium, where character  
options can be found, and the last is the GM section for tweaking rules, creating NPCs, and  
running missions.
 
\subsection{Setup}
                                                  SETUP  

This game makes use of two types of dice, the 20 sided dice (referred to from hereon as a d20)  
and the 6 sided dice (referred to from hereon as the d6). Multiple dice will be referred to in the  
following format - 1 six sided die = 1d6, 2 six sided dice = 2d6, etc. 
 

Sometimes the rules will call for you to roll a 1d3. That is simply a 1d6 with the results halved and  
rounded up (1-2 =1, 3-4=2, 5-6=3).
 

Each player should have at least one 1d20 and a number of d6s. Players will also need a  
character sheet or a piece of paper to write down information, and it might be helpful to have  
paper with a square or hexagonal grid on it (such as graph paper or a pre-prepared battle map)  
since this game makes use of tactical combat. Miniatures are not required to play this game but  
can make combat easier to visualize.
 

One player must play the Game Master (referred to from hereon as the GM). The Game Master  
acts as a referee, storyteller, and arbitrator of rules. They help create the story and narrative for  
the game and play all of the non-player characters (NPCs). For more information on the Game  
Master as well as a list of rules, tips, and tools to use as a Game Master, you can refer to the GM  
guide in the section at the end of this book. The rest of the players will play the role of pilots, or  
characters in that story.  

                                          THE GOLDEN RULE  
When referring to the rules in this book, specific rules override general statements or rules. 
 

For example, making a ranged attack typically takes into account cover. However, a weapon with  
the Seeking keyword ignores cover. In this case, the weapon’s keyword supersedes the more  
general rule.  

                                                 ROUND UP  

                                                                                                              


Always round up in LANCER (to the nearest whole number).  
\subsection{Space and Measurements}
                                 SPACE AND MEASUREMENTS  

This game makes use of measurements in ‘spaces’ for ranges such as movement, weapon  
ranges, etc. Most things in the game are measured in sizes, with size 1 being a square or hex 1  
space wide on each side. By default 1 space = about 10 feet in game. Size indicates the physical  
presence of a mech or other actor on the battlefield. It is measured as a number where size 1 = a  
square or hex measuring 1 on each face. Size can be smaller than 1, such as 1/2, or larger, such  
as 2 or 3. An actor or object is usually as tall vertically as its size, but that’s not always the case.  
Size often does not represent the physical size of an actor, but the space they control around  
them. An actor takes up a square area on the battle map equal to its size, rounded up, with a  
minimum size of 1. For example, a size 1 mech takes up a 1x1 square area and a size 1/2 human  
would also take up a 1x1 square area.  

It’s recommended to use a map and tokens, icons, or miniatures to track actors while playing this  
game for ease of play (though you can certainly play without them). You can easily play this game  
on a tactical grid or hex battle map (as the designers of this game do) or simple measure ranges  
using a standard ruler or measuring tape.  

The scale of space can be changed if the situation needs it - for example, you might decide each  
space is 50’ on each side, or a mile, or something similar. The space that an actor occupies does  
not necessarily indicate its size, but the space it controls around it. Most actors take anywhere  
from a 1x1 to a 3x3 space on the map, with some exceptions.
 

Measure all ranges indicated for weapons, effects, etc as originating from any exterior side of a  
an actor. That means larger actors will have slightly longer range for their weapons or actions. To  
be in range, an actor must be physically inside the range of an effect.
 

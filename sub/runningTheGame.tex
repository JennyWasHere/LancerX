\section{Running the Game}
RUNNING THE GAME  

The following section includes some advice and clarification for helping you actually run the  
game. Generally as the GM it’s not actually your responsibility to know all the rules (that’s what  
this book is for!) but there are some conceits that can be helpful for you to keep in mind.
 

                                            The Golden Rule  

Here it is again for your convenience: When referring to the rules in this book, specific  
statements override general statements. Armor normally reduces all incoming damage, but  
certain tags (AP) and certain weapons or mods (paracausal ammo) can go right through it.
 

                                       A couple good principles  

A good principle to follow as a GM is to try as much as you can to play in reaction to player  
action. Player rolls do ‘double duty’ for you - they determine if a character is successful but also  
give you clues on how to move the story forward. Try to require or ask for rolls in response to  
player initiative, rather than straight up asking for certain rolls. This naturally creates stakes and  
consequences connected to the player in question. 
 

This can be a little tricky with quiet or less proactive groups, in which case it can be useful to  
elicit responses from your group.
 

Eliciting responses isn’t really as complicated as it sounds. It’s very useful for GMs that have  
trouble keeping player attention, or have players that are more hesitant to take action. It can be  
very helpful when a game is stalling or stagnating. Here’s a couple things you can do to get a  
game moving and elicit action from your players:
 
            -   Ask questions. A really simple one. Here’s a couple good examples:  
                     -   What do you think you’re going to do next?  
                     -   How do you/how does your character feel about this?  
                     -   Who’s feeling suspicious here?  
                     -   What do you think is really going on?  
                     -   What’s the way forward from here?   
            -   Address characters, not players. For example, address Chandler, the mech  
                 pilot, instead of Jeff, the player, when you’re talking to them.   
            -    Be descriptive, and try not to describe things in terms of game mechanics first.  
                Ask your players if they want to ‘try climbing that cliff’ instead of ‘make a skill  
                check to climb that cliff’  
            -    Keep things ‘in character’. When players ask NPCs questions or talk to them, try  
                to respond as that NPC and not as yourself. Ask players to try and address each  
                other as their characters as much as possible. Keeping things ‘in-fiction’ will help  
                 keep the game immersive and engaging.  

                                                                                                          
\subsection{Skill Checks}

                                           SKILL CHECKS  

Skill checks in LANCER at always a 10+ (whether they are pilot or mech skill checks). NPCs  
introduce some ways to scale up the difficulty of their skill checks in their profiles, but during the  
course of narrative play, you can also tweak skill checks to offer more or less of a challenge.
 

The first and most important question to ask when deciding whether a skill check is  
required is whether that check is even necessary at all. When the outcome is uncertain,  
important, has clear or relevant stakes, or would lead to an interesting situation (success or fail),  
it’s generally correct to make a skill check. If the outcome is not important, let the players  
automatically succeed.
 

This is especially relevant for tasks that you think might be plot important, simple to accomplish,  
or might stall the plot. You shouldn’t have players make a pilot skill check to see if they know  
plot-important information - just give it to them. 
 

For another example, the players come across a heavy boulder blocking the road. One of the  
players decides to use their mech to push the boulder aside. If doing so (or bypassing the  
boulder) is not actually that important, pushing the boulder isn’t that dangerous or risky, or it’s  
not a particularly big boulder, then don’t even roll - the player just does it. However, it’s probably  
easy to think of a number of reasons a skill check might be required in this situation - the boulder  
could fall and potentially damage a mech, the boulder could roll aside and cause collateral  
damage, or maybe the mech in question is moving the boulder in the middle of an ambush.
 

The second question is what the purpose or goal of the skill check is. This will help set up  
fictional consequences, establish stakes, and make sure you don’t make repeated rolls to  
accomplish the same goal. Players only need to make one roll to accomplish their stated, direct  
goal.
 

If you need to make a skill check, most of the time, the roll should be made without any  
additional accuracy or difficulty required (only that added by the player’s backgrounds, traits, or  
talents, etc), just a flat 10+. However:
 

If you want an easy skill check, you could have the player roll with +1 or +2 bonus accuracy. If  
you’re going to have players make an easy check, you should really ask if they need to even  
make a check at all.
 
If you want a harder than normal skill check, you could have the player roll with +1 or +2  
difficulty.
 
If you want a very hard check, you could have the player roll with +2 or +3 more additional  
difficulty.
 

If the check is so hard that would need to add +4 difficulty or more, just tell the player what  
they’re doing isn’t possible with the approach they are taking and offer them a different  

                                                                                                          


approach. Maybe that boulder is too big to move alone - they need some leverage, or for  
multiple mechs to move it.
 

                                          Helping on skill checks
 

If one or more players want to help on a skill check outside of combat, grant the player making  
the check +1 Accuracy (regardless of number of players helping). The players helping also share  
in the consequences of success or failure.
 

                                              Failing forward
 

Failing forward simply means that narrative should not be predicated on the outcome of skill  
checks, but rather pushed forward by them. You can use complications to push the narrative  
forward even when players fail.  

For example, if your players fail to hack a door, not only do they fail to hack the door, but the  
guards are now alerted to the system and are on their way. Or perhaps they instead spy a vent  
they can enter - a more dangerous but reliable way of getting to their goal. Perhaps they need to  
come back with a piece of specialized equipment that will open the door for sure. They can find  
nearby, but it’s guarded. Or perhaps the door DOES open, but onto an entire corridor full of  
guards.
 

                                         Making repeated checks  

If a task can’t be completed in one roll, set a skill challenge (see the GM tool). Never make more  
than one skill check for the same goal.
 

You can stretch rolls narratively as much as you like. If you want to ‘montage’ or speed through a  
scene, this is a really useful tool. Climbing an entire mountain could be a process of only one hull  
check, for example (if the details aren’t that important). The outcome is what’s important in the  
long run.
 

Remember that if players fail a skill check, they can’t repeat it until they change the  
narrative circumstances. If they fail at lifting the boulder, they can’t do it the same way again -  
they need a different approach.
 

                              Clearly communicate stakes, and commit
 

It’s very important to clearly communicate what’s at stake when making skill checks. You can  
do this naturalistically.
 
	        “Hey Chandler, I see you’re going to use your mech to lift this boulder. Just know, the  
boulder is really heavy and dropping it could probably do a whole lot of damage to whatever’s  
underneath it.”
 

                                                                                                            


Allow players to ‘back out’ of rolls once the consequences are made clear to them. It’s totally  
fine for players to change their minds once they see how risky things will be. That way, the roll  
feels fair and you can easily renegotiate the roll with your player if they want an easier or less  
risky approach.
 

Commit to the consequences of rolls. If Chandler drops that boulder, you can be damned sure  
his mech is taking damage. Consistency is important, and if you’ve already clearly  
communicated that he might take damage, then commit to it.
 
\subsection{The Session}
                                            THE SESSION  

Here’s some good basic rules, terms, and things to keep in mind during a typical session:
 

                                         Rests and Full Repairs
 

Players can take a rest whenever (it takes about an hour) as long as they have the time and  
space. During a rest their can repair their mechs by spending repairs, repair destroyed weapons  
or systems, and clear all heat and statuses from their mech. Some talents, systems, etc, only  
activate on a rest, like the Grease Monkey’s talents.
 

Players can only full repair by taking ten hours. Imagine a full repair like a total reset - they can  
re-build their mech, refresh their repair cap, clear their critical and heat gauges, clear all  
conditions on their mech, heal to full, gain all [limited] systems and weapons back, and regain  
Core Power.
 

Full repairs are more under your control, and so access to them will set the pace for your game.  
Remember though that the GM agenda is not to punish the players - if they need to full repair  
badly, give them a spot they can do it, or else offer Power at a Cost (the downtime action).
 

                                                Core Power  

Character’s mechs have a core power section, which is a box they can check. They either have  
it or they don’t (you can’t ‘save it up’). All mechs regain core power when they full repair. Core  
power can be spent to activate the very powerful CORE systems, which are a ‘one and done’  
sort of deal, only typically activated once per mission.
 

If players want more core power you can use it as a reward or grant it to them as a boon in  
certain situations. It’s always up to you as a GM, except when players take a full repair (they  
always get it back). Granting players more core power lets them use their CORE systems again  
(very powerful abilities), so keep that in mind.
 

                                             Balancing fights  

                                                                                                          


Generally players should be able to complete one or two encounters before needing to rest and  
repair, and 3-4 encounters before needing to full repair. This is assuming that encounters are  
reasonably challenging for the players, and don’t take this number as a hard, inflexible number.  
You should always prepare combats for players with the expectation that things might either go  
very well or very badly for them and your plans (or the character’s plans!) might need to change.
 

Don’t withhold the opportunity to full repair or rest over the idea of verisimilitude.
 

                                          Mech Combat Length  

Mech Combat starts when hostile action is taken by any player or non-player character. It’s  
played out according to the turn/round based combat rules found in the main section of this  
book, and it ends when one or all opposing sides are subdued, surrender, flee, or are completely  
destroyed.
 

If players have overwhelmingly won a combat and there is little remaining threat (for example,  
there is only one weak enemy left for four players) it is very possible to simple declare combat  
ended and decide the outcome of the remaining enemies narratively.
 

Certain FRAME systems and modules remain active ‘until the end of the current scene’. All this  
means is the module remains active until the scene in which they were activated is completely  
over. Otherwise, if activated outside of combat (or if you need a narrative timer), FRAME systems  
are very taxing on a mech’s power systems, and typically only remain active for about 15-30  
minutes.
 

                                 Leveling Up and Rewarding Players.  

Players should generally level up (get one license level) once per mission, after completing that  
mission. You can tweak that however you wish, especially if the mission is very long or odious.
 

By default, LANCER deals with player rewards entirely through the leveling system. When a  
player levels up, it is assumed they have amassed enough currency, reputation, connections, etc  
to buy access to the next license level that they get on leveling. Anything else a pilot could buy  
or get their hands on, they should generally be able to just buy it outright (no need to track  
currency), or else make some pilot skill checks to get their hands on it through graft, negotiate,  
connections, or bartering.
 

However, the following section presents rewards you could give out to players as incentives for  
part of a mission, completing certain tasks, or satisfying certain requirements. It’s up to you how  
heavily you want to use these in your game or lean on them to hook your players.
 

1. Use Manna  
You can use the Manna system (see the ‘Changing Core Assumptions’ section), which adds a  
currency system to the game. You can track manna for items, or even use it to replace the  

                                                                                                          


leveling system, in which case players no longer gain License Points when leveling up, but must  
buy them.
 

You check your slate again, not sure you read the glowing number correctly, sure that you added  
another zero on accident. No. It’s all there, all those commas and zeros. You’re rich in Manna,  
fabulously rich. In the zero-G of realspace travel, your stomach turning is both a physical and  
mental thing. You whoop, your cry of joy mingling with the cheers of your squadmates, as their  
slates and subdermals ping, notifying them of a successful transfer of funds.   

With this Manna, maybe you can finally get Boss Kozta’s goons off your back. Maybe you can  
even take back what he took from you. How much did a proper set of STAMPEDE cannons cost  
on the Horus-net again?...   

2. Grant Reserves  
Grant players pilot gear, vehicles, or other useful material that they can use for reserves. For  
example, players acquire a useful vehicle, an enormous drill, blackmail on a politician, insider  
information on a rebel general, or a new hardsuit. They might become friendly with the local rebel  
group, or the hard-bitten mercenary at the bar, or the socialite who controls the cash flow on the  
space station.
 

\_\_\_\_\_\_\_\_\_\_\_\_\_
 

The Administrator, as she promised, returned. You and your small band greet her at the makeshift  
spaceport, an old marble quarry with a rickety scaffold tower overlooking it to sight ships  
approaching the recessed landing zone.   

“We’ve waited years,” you say, speaking first. You’re decades older now, but the Administrator  
doesn’t look a day older than when she left. Your heart, your soul. You think of your children’s  
mother, out even now in the timberfields.   

“The ship is yours,” the Administrator says. She tosses you her slate. “Access, flight plans,  
transponder codes. It’s all on there. The NHP is tuned to you, already. I’ve been teaching it.”   

“The ship is mine,” you repeat. A reward, of a kind…  

\_\_\_\_\_\_\_\_\_\_
 

Over your chassis’ omni, a cracking voice.  

“That did it! The hardlight wall is down! All units, push forward -- Green squadron, Red squadron,  
lay some fire down!”   

                                                                                                              


You lay back in your crash couch, the gimballed cockpit of your chassis adjusting for the move.  
You did it. Your squad keys in over the local band, cheering.   

You made a breach. Already, over the wide band, the battlescape was alight. Reinforcements  
were pouring in through the breach. Somehow, impossibly, the battle had turned in your favor.   

“Gold squadron,” the Legion’s level voice.   

“Go ahead, Command.”  

“Good work, Gold squadron. Report back to the waypoint marked on your HUD. Your job is done  
for the day: all scenario probabilities report total success from this point on.”    
 

4. Grant Skill Points  
You can directly grant players pilot skill points to spend on pilot skills (+2 at a time). A player that  
has been learning to pilot a starship could easily be rewarded +2 to Get Somewhere Fast after a  
mission to represent their diligence and study. Doing so increases player power levels, so use  
this reward carefully.
 

5. Reward a Unique or Restricted system or weapon  
Rewarding your players with items that are unique, exotic, or otherwise restricted from their usual  
requisition pool is the closest thing in Lancer to magical or wondrous items typically found in  
fantasy tabletop RPGs. 
 

The easiest option is to reward players with a weapon or system from a license they do not  
or cannot have access to. The weapon or system can only be used for one mission (think of it  
like a ‘rental’), then they lose access to it. 
 

Fatigued like you’ve never known, you crash down into your bunk, not even bothering to get all  
the way out of your flight suit. You kick your boots off, toss your insulating hood onto the floor of  
your cabin. You’ll get it later, firs you need to rest.   

“Hey flyboy, Cap’s got something for you.”   

The crewman’s bark wakes you not minutes later. You sit up, groaning, and see with a start that  
the crewman is accompanied by the ship’s XO and head motor pool engineer. You snap a salute,  
which they wave off.   

“You did good out there. Still more work to do. Motor?” The XO says in his characteristic gruff  
voice. The ship’s head engineer steps forward and presses his personal slate into your hand.   

“Anything you want, kid. Just learn it first before I have to hoze you out of your cockpit.”   

                                                                                                           


You scroll through the list, previously locked licenses unlocked and waiting your requisition. The  
fatigue disappears, replaced only by excitement…  

6. Grant Exotic Tech  
Drawing up exotic or truly unique systems or weapons is a bit more of a process. We  
recommend adapting your exotic system or weapon to the narrative you’re running. We will  
eventually include a table of exotic weapon/system types here to get you started; official Lancer  
narratives will feature their own exotic weapons and systems. 
 

Exotic tech refers to a particular type of mech system or weapon which is typically unlicensed,  
unsanctioned, experimental, or non-human in origin. Due to its nature, exotic tech cannot be re- 
printed when a mech is destroyed, and is lost permanently unless the weapon or system itself  
can be salvaged.
 

Exotic tech can be a way for GMs to offer physical rewards to players without directly giving  
them more license or talent points.
 

 It follows the following rules and conventions:
 
    -    Installing or uninstalling a system or weapon with the Exotic tag requires you take a full  
         repair  
    -    Exotic tech is typically more powerful than comparable tech  
    -    A weapon or system with the exotic tag cannot be re-printed with your mech should it  
         be destroyed, but must be physically re-acquired  

Here’s a couple examples of Exotic tech for your use. We’ll include a short table in a future  
update. These are not particularly balanced in any way, but might give you a general idea of what  
to look for.
 

Miniaturized Nuclear Missile  
Your mech is equipped with the latest in thermonuclear technology, typically reserved for ship-to- 
ship combat.  

Superheavy Exotic Launcher
 
Range 50
 
Limited (1)
 
Blast 20
 
10d6 explosive damage + 10 heat
 

Mechs caught in a blast 40 zone centered on the impact point must pass a systems skill check  
with 2 Difficulty or be immediately shut down. This missile can never be replenished once used.
 

Living Metal  
Your mech has partly biological components of alien origin that automatically crawl over damaged  
parts of your mech and knit them back together, wire by wire.  

                                                                                                             


2 SP
 
Exotic, Unique, Biological
 
Your repair cap increases by 4. Each round, you may spend 1 repair once to heal as an end-of- 
round action.
 

The Chosen of Aun fell, its golden chassis trailing a greasy pall of smoke from its shattered  
cockpit.   

You step forward, you chassis moving as an extension of your own form, ceramoferrous plating  
ticking and cooling as you vent your chassis’ heat tax. The battle has moved on, ignoring the end  
of your desperate, decisive single combat.    

“No signs of life,” your NHP whispers in your aural. “I see incredible tachyon bleedout,  
ontological stuttering.” She pauses. “There’s something else in there, sir. Be careful. I cannot see  
it. Raise your shield.”   

You follow her suggestion, hefting your stasis wall.   

The shattered Chosen twitches, its tons of ruined machine-mass rattling in death. A light burns  
from the belching smoke.   

“That is it, there, in the void I cannot see. What is it?” Your NHP whispers.   

A steady wind tugs the smoke away, and you see it.   

A golden disc, broad and hammered, unadorned. A light like the sun streams from behind it no  
matter which way you view it from.   

“It… is perfect,” you whisper back. You reach out a delicate manipulator, grab the disc and pull it  
towards your chassis. You feel the sudden attunement, the connection. Yours, so long as you  
keep it.   

But what does it do?...  

7. Reward Talent Points
 
Talent points can be directly awarded to players (as they are not necessarily locked to level) and  
spent as normal. The world of LANCER grants easy explanation for this sudden burst of  
instantaneous talent - there are a great number of neurological implants available for purchase  
from military and civilian sources.
 

                                                                                                               


Granting players increased numbers of talent points can be very powerful, so you should use this  
option sparingly.
 
\subsection{Moving Forward}
                                       MOVING FORWARD  

We’ve reviewed some of the intricacies of skill checks, we’ve talked about hooks, and we’ve  
talked about rewarding and pacing players. The following sections will help you to further  
customize your game by adding NPCs and changing, adding or adjusting some core rules.  
The GM toolkit includes some additional resources for fleshing out your game as well as rules  
for changing some of the core conceits of the game and adding more complicated pilot play. The  
NPC section includes statistics for creating non-player characters for use in combat, and some  
tips on creating NPC characters for narrative play.
 
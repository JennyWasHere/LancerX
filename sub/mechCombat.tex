\section{Mech Combat}
MECH COMBAT  

It’s entirely possible to play through a session of LANCER without even touching mech combat.  
Some groups may prefer a more role-play, politically heavy game in which most of the fights are  
decided with skill checks.
 

However, Lancers are people with a particular talent, and it’s almost inevitable that talent will be  
put to good use during a mission. Sometimes, you want combat to be more in-depth, and each  
decision to matter more. You want your skills and ingenuity at building and fighting with a mech  
put to good use. That’s the signal that its the perfect time to get into Mech Combat.  

Unlike narrative play, mech combat is tactical, and turn based. As the name implies, these rules  
are only used for combat, and probably when mechs are involved. You can certainly run pilot on  
pilot fights using mech combat, but the options are far less interesting. 
 

Here are some key differences between running combat narratively (with skill checks) and mech  
combat:
 
        	- Instead of using your pilot skills to narratively resolve combat actions, you must take  
        turns, and you have limited options on your turns to attack, move, and activate  
         components of your mech.
 
         - Your pilot and mech make attack rolls, adding grit, to fight opponents. This might mean  
        that even though your pilot has a higher combat related skill (assault, for example), they  
         must use a lower value. This is because mech combat is at much higher stakes, and at a  
         much higher scale! Only your pilot’s direct experience fighting in a mech (and their four  
         mech skills) is going to help them succeed.
 

                                                  Use a map  

It is recommended you use a map of some sorts, and draw out or use items such as miniatures,  
tokens, etc to track the position of players during a fight. A grid (hex or square) will also help  
immensely. You can run LANCER without a grid, using each space 1 to 1 for inches or cm on a  
ruler and measure directly, but it will be far less consistent.  

                                              Starting Combat  

To start combat, the GM merely needs to declare that it has been initiated. Hostile intent, such  
as firing a weapon at a target, attempting to grapple them, or charging a target will typically  
automatically initiate combat. Establish where the various NPCs and players are when combat  
starts before you start rolling or picking turn order, it will help visualize things better.  

\subsection{The Turn}
                                                THE TURN  

                                                                                                              


During combat, players always take the very first turn. One player or friendly NPC (nominated  
by all players) gets to act first. If the players can’t agree, the GM chooses. After that player  
finishes their turn, the GM may activate a hostile, GM-controlled NPC, allowing them to take a  
turn. Each NPC can usually only be activated once, unless they have special traits. The player  
that previously acted then nominates a player or friendly NPC to act next, and so on. Each  
actor gets 1 turn in a round, alternating between players and hostile NPCs, with players each  
choosing the next player or friendly NPC to act.
 

If there are only actors of once side left, the remaining actors take their turns in any order. After  
all actors have completed a turn, this constitutes 1 round. The round then begins again,  
alternating, so if one side ended the last round, the other side starts the new round. This may, for  
example, mean that hostile NPCs take the first turn in the new round if the players outnumber  
them.  

On a turn, players and NPCs can perform a move, and either two quick actions or one full  
action, with no duplicate actions allowed. Players can overcharge their mechs to gain an extra  
quick action at the cost of heat, and all actors can also take any number of Free Actions or  
reactions.
 

MOVE - A player can move their character up to their full movement speed.
 
QUICK ACTION - A quick action represents an action that takes a few moments, such as  
quickly firing a weapon, using a system, or moving a little further
 
FULL ACTION - A full action represents an action that takes your full attention, such as a  
sustained barrage of fire, or field repairing your mech
 
FREE ACTION - A free action can be made at any point during your turn, but only on your turn. It  
doesn’t count as a quick or full action, so you can still make those as normal. Free actions can  
also be used to make a duplicate action (for example, a free action could allow you to boost if  
you have already made that action). You only get free actions if some part of your character  
grants you them.
 
REACTION - Reactions are special moves that can be made out of turn order in response to  
incoming attacks, movement, or other prompts. You can make each reaction only a specified  
number of times per round, but take as many overall as you want. By default, mechs have two  
reactions they can take once a round: brace, and overwatch but they may gain more from  
systems or talents. Reactions resolve before the triggering action completes by default, but  
some may resolve after.
 

                                                      PILOTS
 
On foot, a pilot has the following statistics in mech combat:
 
	        HP: 6 + grit
 
	        Evasion: 10
 
         E-defense: 10
 
	        Armor: 0
 
	        Size: 1/2
 

                                                                                                                  


	        Speed: 4
 

These statistics might change depending on the gear and armor a pilot brings with them.
 

Pilot weapons and armor are at a scale that they can’t be relied on to take down mechs - and  
mech weapons are at a scale that they normally completely pulverize a pilot-scale foe. The  
following rules apply to pilots (some of these refer to mech rules later in this section):  
     -   Pilots have the biological tag. They are immune to Tech actions (even beneficial ones),  
         though they can still be targeted by electronic systems such as drones or smart weapons.  
         If a pilot would take Heat, they instead take an equivalent amount of energy damage.  
     -   When a pilot is called on to make a mech skill check, they use Grit instead of the required  
         statistic  
     -   Pilots can’t aid a mech, give, or receive any bonuses that would apply to mech-sized  
         weapons (such as from Talents)  
     -   Pilots and pilot weapons and gear don’t benefit from Talents  
     -   Pilots can’t cause a mech to become engaged and don’t provide obstructions to mechs no  
         matter the size.  

It is possible for a pilot, with enough experience, to gain enough technology and experience to  
become capable of fighting on nearly even terms with some mechs, but such pilots are usually  
stuff of legend.  

                                                Pilot, Mech, and AI  

As components of the same character, pilots and mechs share the same move and actions. You  
can split them up if you so choose. If you want to use a quick action to skirmish with your mech,  
use another quick action to dismount, then use your move to run to cover as your pilot, you can  
absolutely do so.  

A mech needs to be piloted for you to take actions with it, with the pilot physically present inside  
the cockpit, unless that mech has the AI property. If your mech has the AI property, at the start of  
your turn you can choose to turn your controls over to your AI. If you do so, your pilot can no  
longer take actions or reactions with your mech until the start of your next turn, but your mech  
gets its own set of actions and reactions, freeing you up to take normal action as a pilot. However,  
your AI cannot benefit from any of your talents while it pilots your mech.  

                                                                                                                      

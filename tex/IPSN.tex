\section{IPS-Northstar}

  IPS-NORTHSTAR   

                              “YOUR FRIEND IN AN UNFRIENDLY SEA”  

IPS-Northstar is the child company born from the merger of civilian cargo lines  
Interplanetary Shipping and Northstar.  

Space piracy and rogue state actors remain the greatest threat to interstellar shipping lines,  
costing ship owners trillions in Manna and countless more in their local currencies. After incurring  
tremendous capital losses due to piracy, IPS and Northstar decided to announce a collaborative  
merger in order to ensure the safety of all civilian and corporate shipping.   

Initially utilizing late-model GMS line mechs, the new IPS-Northstar corporation quickly  
developed their own makes and models of versatile, durable, modular mech chassis that mount  
weapon and engineering systems in equal measure. IPS-Northstar mechs are a good choice for  
pilots who want a tough chassis that’s built for close quarters and melee combat where breaching  
a ship hull might be a possibility. IPS-N mech chassis are built sturdy, meant to take as much  
damage as they can deal, and then some.    

IPS-N is most closely associated with The Albatross, a Cosmopolitan anti-piracy/peacekeeping  
force known across the galaxy for their long and storied history of humanitarian intervention. IPS- 
N supports the Albatross materially, providing them with chassis, ships, and cutting edge IPS-N  
tech -- the relationship between the two groups is mutually beneficial, and IPS-N makes a point  
to emphasize their close relationship to the Albatross in marketing campaigns and PR materiels.   

IPS-N Mech FRAMEs: 
 
IPS-N DRAKE (Heavy Assault)  
IPS-N BLACKBEARD (Melee)  
IPS-N TORTUGA (CQB)  
IPS-N NELSON (Mobile Melee)  
IPS-N LANCASTER (Repair/Support)  
IPS-N VLAD (Special Assault)  
IPS-N RALEIGH (Line mech)  

                                                                                                          
\subsection{IPS-N Pilot Gear}

                                                IPS-N Pilot Gear  

  Name                     Tags                                      Range            Damage                   Rarity 

  “Peacekeeper” R35        Limited (4), Sidearm                      3                2 explosive              1 

 Hackiron                  Reliable 1, Inaccurate                   Threat 1          3 kinetic                2 

 Ripjack                   -                                        Threat 1,         2 kinetic                3 
                                                                    Thrown 4 

 Siege Hammer              Loading, AP, *                           Threat 1          4 kinetic                3 

 Prism Gun                 Inaccurate                                Cone 3           2 energy                 4 

                                                     Pilot Weapons  

*See Entry
 

Hackiron  
This blade uses an ultra-dense carbon polymer to magnify the brute strength from hard suits to crude but  

brutal effect. Made popular by security forces and pirates both, the hackiron-type saw sword was put into  
heavy use in early micro/null gravity melee combat where mass is not always a necessary consideration,  
gravity can fluctuate, and it is likely that any combat will be powered combat.  

A pilot must be wearing armor with the armor tag to even wield this weapon  

“Peacekeeper” R35  
IPS-N’s Peacekeeper model R35 is a popular option from their ARGONAUT line of luxury/collector goods.  

Handcrafted by IPS-N gunsmith artisans, the Peacekeeper model R35 is fashioned after ancient pre- 
collapse frontier weaponry: functional, reliable, and simple in its action, the P-R35 hides its expertly crafted  
interior elements under its simple matte finish. It is a statement weapon, functional, that can be chambered  

for heavy slugs, self-propelled microrockets, or single-core hotshots.  

Prism Gun  

A common secondary weapon found among stellar security forces and pirates, a prism gun is a shotgun  
analog, a blend between the kinetic and the exotic that projects a superheated blast of synthetic-crystalline  
nanoflechettes when fired. Ineffective against hardened targets, it is nevertheless a popular choice as a  

finishing option should a hardened targets external armor be breached; when fired into compartments,  
between armor joints, or into confined spaces with soft targets, the prism gun is a terribly deadly weapon.   

Ripjack  
Another standardized utility weapon born from early micro/null-grav combat, ripjacks are employed both as  
a last-result method of locomotion and a distance-closing utility weapon. Offered both as a handheld  

launcher and an integrated hardsuit system, the ripjacks are fired using gaussian acceleration to create as  
light a Newtonian feedback as possible when used in null-grav.     

                                                                                                                       


The ripjack can be used as a Thrown 4 weapon against a human sized, non-mech target in range,  
immobilizing them on hit instead of dealing damage until the end of next round. It can also be used as a  

regular grappling hook.
 

Siege Hammer  

When faced with sealed bulkheads in O   rich environments, boarding teams needed a way to tear down  
                                             2  

doors without potentially catastrophic explosions -- enter the siege hammer, or Nutcracker, as it became  
known to forces that employed it. Requiring two hands to wield effectively, and held like a battering ram, it  

cocks and primes a superheated cylinder of ultra-hard alloy that fires on activation to punch through even  
the thickest armor.  

While wielding this weapon, your pilot cannot take the boost action.
 

                                                 Clothing and Armor  

 Name                                  Tags          Bonuses                        Armor     Evasion/      Sp     Rarit 
                                                                                              E-            d      y 
                                                                                              defense 

 Combat Webbing                        Upgrade       Bonus weapon                    -        -             -      1 

  WAYLAND Mobile Cuirass               Armor         +3 HP, Increased Mobility      0         10/10         4      2 

 DURENDAL Mobile Cuirass               Armor         +3 HP, Unarmed attack          1         8/8           4      2 

 CALADBOLG Mobile Cuirass              Armor         +5 HP                          2         7/7           3      4 

 Goliath Weave                         Upgrade       Bonus HP, increased            -         -             -      3 
                                                    strength 

*See Entry
 

CALADBOLG Mobile Cuirass  

The CALADBOLG-MC is a heavy tactical hardsuit, with plated armor coverings, an integrated powered  
exoskeleton, and hardpoints for systems, integrated weapons, external power sources, and high-fidelity  
chassis links; it is fully sealed, modular, and hardened from radiation and other HAZMAT threats. The  

CALADBOLG-MC boasts one of the highest survival rates of all hardsuits in the field and is known for its  
reliability and durability. No fancy tricks with this one, the CALADBOLG is heavy, simple, and does its job.     

Combat Webbing  
A series of straps and combat gear that allows the wearer to bring an extra weapon into battle past the 1-2  

normally allowed. If you like pouches, this one’s for you. 
 

DURENDAL Mobile Cuirass  

IPS-N’s DURENDAL-MC hardsuit is a popular choice for frontline marines, security forces, and mercenaries  
who expect to see high-fragmentation, high-kinetic combat. Built on top of a base, sealed suit of mixed  
anti-ballistic weave and insulated cordage, the DURENDAL features sloped core and dorsal plating, leg and  

                                                                                                                       


groin plating, brachial plating with mounts for integrated systems and weapons, compatibility with any IPS- 
N helm, and total powered-system integration. the DURENDAL is IPS-N’s signature frontline hardsuit, and  
through its modularity it is well-suited to space and terrestrial movement.        

Even while unarmed, a pilot wearing a Durendal counts as having a sidearm (1 kinetic damage, sidearm,  
range 3) and a light melee weapon (1 kinetic damage, sidearm, threat 1). They don’t count against  

weapons a pilot can take with them on a mission.
 

Goliath weave  

Goliath Weave is an innovative new approach to the standard IPS-N exoskeleton. A product of IPS-N’s R\&D  
campus on Vela, Goliath Weave is a woven, miniaturized, fiber-adjacent mesh augmentation that can be  
layered over extant structures in order to boost physical durability and strength.  

A pilot must be wearing armor to benefit from this enhancement. While wearing this armor, a pilot gains +3  
HP. They can lift, drag, or push items other than mechs up to size 1.
 

WAYLAND Mobile Cuirass  
The WAYLAND-MC builds off of IPS-N’s DURENDAL platform, stripping some of the integrated weapons  

and systems and replacing them with a system of integrated grapples and an impulse pack, which allows  
for limited periods of thrust-assisted mobility.   

The WAYLAND-MC has an integrated grapple system and short-burst jetpack, allowing a pilot to fly their  
speed when they take the boost action (the pilot must land by the end of their turn or fall). The grapple has  
a range of 4 and can also be used to cling to walls, snag on to items, etc.
 

                                                      Miscellaneous 

  Name                          Tags         Description                                                            Rarity 

  K-Cal Wafer                   Upgrade      Provides a full day’s worth of sustenance in compact form.             1 
                                             Nobody can vouch for the taste. 

  Null-Rad Caps                 Upgrade     Anti-radiation medication for extended interstellar trips. If           1 
                                            taken in concentrated form, a pilot can pass through areas  
                                             of high or even deadly radiation for around an hour, with  
                                             relative safety. 

  Lifesaver                     Upgrade     A personal transponder and vacuum movement device, in                   1 
                                             case you get spaced. The lifesaver compacts to a tiny  
                                             pack that can be strapped on the body, provides good  
                                             EVA movement, and once deployed has power for several  
                                             days. 

  Vascular Stabilizers          Gear        A custom drug cocktail that helps the human body deal                   2 
                                            with combat-rated gee forces; ensures blood movement to  
                                             brain, mixed with a suite of anti-nausea and anti-fatigue  
                                             drugs. When you take them, until the end of the current  
                                             challenge, you can re-roll any Down and Out check you  
                                             make, though you must choose the second result. 

                                                                                                                          


*See Entry
 

                                                                                                                          \subsection{IPS-N Core Bonuses}


                                        IPS-N CORE BONUSES  

When you choose a core bonus every 3 license levels, you can pick a bonus from this list as long  
as you have at least 3 license levels in IPS-N licenses for each IPS-N bonus you have. For  
example, if you have 6 points in IPS-N licenses, you could take up to 2 bonuses. IPS-N bonuses  
are focused on increasing durability, survivability, and melee combat.
 

These bonuses apply to any mech you create each time you create it, and each can be chosen  
only once.
 

BRIAREOS frame reinforcement  
The BRIAREOS is the newest development in IPS-N’s line of near-fail frame upgrades, templates designed  

to maximize what pilots can get out of frames before the need for a reprint or catastrophic failure.  
BRIAREOS fabricates a frame that is superlight, woven throughout with Goliath Weave meshing, to  

increase the resilience of all inorganic components.   

Your mech has resistance to all damage when it has 1 or less points of structure remaining.
 

FOMORIAN frame reinforcement  

The FOMORIAN is an up-scaled version of the stock IPS-N template, following guidelines suggested by  
long-haul Cosmopolitans in need of robust micro/macro-impact protection, as well as additional brachial/ 
manipulator needs.  

Your mech goes up in size to the next increment (1/2 > 1 > 2 > 3), up to size 3 maximum. It gains  
+1 threat with melee weapons, and cannot be knocked prone, pulled, or knocked back by  
targets smaller than itself (as part of a grapple, for example).
 

GYGES frame reinforcement  

Gyges is a frame upgrade built for combat, with finely tuned stabilizers and a robust suite of targeting  
software and hardware included at-fabrication.   

Your mech gains +1 accuracy on all hull checks
 

Reinforced Frame  

The addition of redundant shock-absorption systems increases the survivability of a pilot in combat, flight,  
and kinetic situations.   

Your mech gains +6 HP
 

Sloped Plating  

A simple, common enough option among pilots with the necessary licensing, IPS-N’s armor integration  
fabrication reduces the plating coverage gaps in stock systems by a significant percentage.   

Your mech gains +1 armor, up to the maximum (+4).
 

                                                                                                                 


Titanomachy Mesh   
A doubled overlay of Goliath Weave, fabricated in identified stress points, as well as beefed-up  

specifications, dramatically improve the baseline functionality of this mech.   

1/round when you successfully ram or grapple a mech, you can make an additional ram or  
grapple action as a free action. When you knock targets back with a melee attack, you knock  
them back 1 additional space.
 

                                                                                                                     


                                                                                                                         

                                                                                                                       
\subsection{IPS-N Blackbeard}

                                         IPS-N BLACKBEARD 
 

The IPS-N BLACKBEARD is IPS-N’s solution for an aggressive, front-facing, preemptive anti-piracy  
platform. The BLACKBEARD license range is built for environments where combustible kinetic weaponry is  

either useless, too dangerous, or would prompt unnecessary collateral damage. Its distinctive, slim frame is  
a evocative of its speed and reduces its radar profile, making it hard to track and harder still to hit. The  
BLACKBEARD platform has been split into two model lines, the IPS-N/BB-L, which is the standard  

production line model, and the IPS-N/BB-Sk, a prototype limited print run of BB models purpose-built to  
contain IPS-N’s SEKHMET NHP platform. 
 
                                                     License:
 
I. Synthetic Muscle Netting, Chain Axe
 
II. BLACKBEARD FRAME, Flechette Launcher, Nanocarbon Sword
 
III. Reinforced Grapples, SEKHMET class NHP
 

                                                BLACKBEARD 

  HP: 12          Evasion: 8                             Speed: 5            Heat Cap: 4        Sensors: 5 

  Armor: 1        E-Defense: 6                           Size: 1             Repair Cap: 4      Tech Attack: -2 

                                                     TRAITS: 

  Cable Grapple: The Blackbeard can initiate grapples up to range 5 away. If it successfully grapples its  
  target, the Blackbeard is immediately pulled adjacent to its target by the straightest path possible (if it  
  can’t move adjacent to its target, the grapple breaks).
 
  Lock/Kill Subsystem: The Blackbeard can boost and take reactions while grappling.
 
  Exposed reactor: The Blackbeard gets +1 Difficulty on engineering checks 

                                               SYSTEM POINTS: 5 

                                                     MOUNTS: 

  Flex mount                          Main Mount                             Heavy Mount 

                                                  FRAME system 

                                                                                                                


                                                   Assault Grapples  
   The IPS-N branded assault grappling system is a proven, class-leading system rated to handle hauling,  
  supporting, and securing chassis up to Galactic Standard Size 4. Grapple heads are interchangeable  
  and can be swapped for hard or soft targets, electrified, or loaded with codespike systems to  
  incapacitate targets at a distance. 
 

  Active (Requires 1 Core Power): Omni-harpoon  
   Quick Action
 
  This one-shot system fires harpoon-like grapples at any number of targets within line of sight and within  
   range 5. Those targets must pass a hull check with 1 difficulty or be knocked prone and pulled adjacent  
  to your mech, or as far as possible towards your mech without being obstructed. All targets are then  
   immobilized until the end of your next turn 

Synthetic Muscle Netting  

IPS-N’s proprietary Synthetic Muscle Netting is a field-proven augmentation compatible with existing IPS-N  

FRAMEs. A spray-on catalytic/structural enhancement, the SMN system boosts manipulator and  
propulsion subsystems by roughly 25% without impacting the operational life of augmented components.  
The spray-on catalytic also acts as a mild impact-absorption and thermal insulation layer; IPS-N  

recommends pilots only apply the SMN system to interior components and practice frequent cleaning to  
prevent septic-analogous decay.    

2 SP, Unique  
When grappling or ramming, you always count as the same size as your opponent if your  
opponent is larger than you, and larger than your opponent if they are the same size or smaller.  
Your lifting and dragging capacity doubles.
 

Chain Axe  

A simple tactical scale-up of a felling axe, IPS-N’s chain axe is a serrated, powered chainblade hardlinked  
to a chassis’ power core. The teeth of the IPS-N chain axe are tungsten-tipped, hardened to chew through  

hard and soft targets both. It is an effective weapon and utility tool, and is often used by boarding parties to  
make initial breaches in ship and station bulkheads.   

Main Melee
 
Threat 1
 
Reliable 2
 
1d6 damage
 
On a critical hit, your target is Shredded until the end of your current turn
 

Nanocarbon Sword  

IPS-N’s nanocarbon sword is a new spin on an old essential. Embedded nanosensors along the length of  

the blade capture a full spectrum of data while in use, recording to cloud-based Omninet storage banks for  
after-action review. Live feedback is relayed to the user, interpreted by their equipped sensor suite, and  
real-time adjustments are made to the molecular composition of the blade edge.  

Heavy Melee
 

                                                                                                                      


Reliable 3
 
Threat 2
 
1d6+4 kinetic damage
 

Flechette Launcher  
The IPS-N Flechette Launcher utilizes a hive-analogous construction to project a total soft target kill zone in  

a dome around the user, denying personnel the opportunity to engage in aggressive infantry-tier actions.   

Auxiliary CQB
 
Burst 1
 
1 Kinetic Damage
 
This weapon deals 3 damage instead of 1 against grappled targets or targets with the biological  
tag.
 

Reinforced Grapples  
2 SP  
Grapple movement: Once a turn, your mech can use this grapple when it makes a regular move,  
allowing it to Fly as long as it moves in a straight line and there is a clear path. It must end its  
move on an object or surface or fall, but can grab on to that surface (even vertical or overhanging)  
as long as it remains immobile. If it’s knocked prone or knocked back while grabbing onto a  
surface this way, it falls.  
Drag Down: These grapples can also be used to target another actor within range 5 as a quick  
action. Make a contested hull check with your target. The loser is knocked prone.
 

SEKHMET-class NHP  

The IPS-N SEKHMET Co-pilot is ready to be your First Mate! SEKHMET comes standard with remote,  
Omninet, IR tag, and voice control systems and is fully versed in all current and legacy IPS-N mech cores.  

Your own SEKHMET system will learn with you, and should the worst happen, will continue as you would,  
running an emulated neural net doppelgänger to control your IPS-N chassis until forced or voluntary  

shutdown.   

SEKHMET-class systems tend to have aggressive attitudes and dark sense of humor; pilots often like to call  

them a berserker system, a dangerous NHP that values combat efficacy over its pilot’s well being.  

3 SP, Unique  
AI  
Your mech gains the AI property. In addition, gain the SEKHMET protocol:
 

SEKHMET protocol  
Protocol
 
        • All melee Critical Hits do an additional +1d6 bonus damage  
        •  You can make a skirmish action using only melee weapons as a Free Action at any point  
          during your turn.
 

                                                                                                                  


While active, you lose direct control of your mech. Your mech uses all available actions and  
movement to first attempt to get into melee range of the closest target (friend or foe!) and then  
attack them using all weapons. If your mech isn’t in melee range of a target, it attempts to use all  
actions to get into melee range, even if it could still fire a ranged weapon using those actions.  
You can decide to overcharge your mech or not, but if you do, it uses the overcharge action for  
the same purposes.
 

To end this protocol, you must pass a successful engineering check at the start of your turn.  
Otherwise, this protocol will continue until your mech is destroyed. Death or incapacitation of the  
pilot will not stop it.
 

                                                                                                                       
\subsection{IPS-N Drake}

                                                                                                                                 
                                                    IPS-N DRAKE  

The IPS-N DRAKE is the backbone of any proactive trade security/anti-piracy force and represents the  
manufacturers first foray into military-grade mechs. It is a massive frame, simian in appearance, built  
around a single-cast bulkhead sloped and reinforced to handle sustained incoming and outgoing fire.    

A dense heavily armored chassis, the standard IPS-N DRAKE fleet license includes a high-velocity, high- 
fragment assault cannon for suppressing and overwhelming their targets, and a heavy kinetic/ablative  

barrier shield for defense. More advanced models feature scaled-up weaponry and armor, including the  
notorious multi-barrel Leviathan cannon.  

                                                                                                                           


                                                     License:
 
I. Assault Cannon, Concussion Missiles
 
II. DRAKE FRAME, Aegis Shield Generator, IPS-N Argonaut Shield
 
III. Portable Bunker, Leviathan Heavy Assault Cannon
 

                                                     DRAKE 

  HP: 8           Evasion: 6                             Speed: 3            Heat Cap: 5        Sensors: 10 

  Armor: 3        E-Defense: 6                           Size: 2             Repair Cap: 4      Tech Attack:  
                                                                                                +0 

                                                     TRAITS: 

  Heavy Frame: The Drake cannot be knocked back or prone by actors smaller than itself
 
  Blast Plating: The Drake has resistance to damage from blast, line, and cone attacks
 
  Guardian: Adjacent allied mechs can use the Drake for Light Cover
 
  Slow: The Drake has +1 Difficulty on agility checks 

                                               SYSTEM POINTS: 5 

                                                    MOUNTS: 

  Main Mount                         Main Mount                              Heavy Mount 

                                          CORE SYSTEM: FORTRESS 

  Active (requires 1 Core Power):  
  Protocol
 
  When you activate this protocol, you plant your shield and deploy stabilizers, becoming more like a  
  fortified emplacement than a mech. While this system is active, your mech is immobilized. Two line 2  
  sections of heavy cover unfold, drawn from your mech in any direction. Your mech grants and benefits  
  from heavy cover for allied mechs while this system is active and also grants any allies that benefit from  
  this cover its immunity to knockback, prone, and resistance to blast, line, and cone attacks. This  
  system can be deactivated at the start of your turn but cannot be reactivated without more core power. 

Assault Cannon  

The IPS-N assault cannon is a deep-cooled autocannon, able to be fielded as a fixed weapon or  
manipulator-compatible platform. This autocannon can be fed by box-magazine or belt, is simple in its  

functionality, and is a mainstay among IPS-N chassis fleet orders.   

Main Cannon
 
1 heat (self), Overcharged
 
Range 8
 
1d6+2 Kinetic Damage
 

                                                                                                                


Concussion Missiles   
Main Launcher
 
Range 5
 
Knockback 2
 
1d3 explosive damage
 
Targets struck by these missiles must pass a hull check or become impaired until the end of their  
next turn.
 

Aegis Shield Generator  

The Aegis is a portable electromagnetic shield generator, a way to establish a momentary safezone to  

withstand an incoming bombardment or environmental hazard.     

2 SP, Unique, Limited (1)  
Shield, Deployable, Quick Action  
Once planted in a free adjacent space, this size 1 generator creates a burst 2 zone around it until  
the end of the current scene. All allied targets at least partly covered by the zone gain +1 armor  
(up to 4). The generator has 10 HP but benefits from its own armor bonus, and deactivates once  
used up.
 

IPS-N Argonaut Shield  

2 SP, Quick Action  
As a quick action, this heavy over-arm shield can be used to protect an adjacent actor from  
incoming fire, giving them resistance to all damage as long as they stay adjacent to you.  
However, your mech takes the half that was resisted. The effect breaks if they break adjacency,  
and you must repeat this action to regain the effect.
 

Portable Bunker
 

A simple deployable, the “Portable Bunker” is actually a series of unfolding single-use printer sheets: flat- 

pack pouches of inert non-newtonian fluid that, when deployed, are triggered into a rigid structure capable  
of withstanding incredible force.   

2 SP, Limited (1)  
Deployable, Quick Action
 
To activate this system, choose a clear 4x4 space adjacent to you and take a quick action. At the  
start of your next turn, this system unfolds into a fortified emplacement that grants heavy cover  
to anyone within the area from all directions, as long as they are fully covered by the area. Actors  
inside also have resistance to damage from blast, line, and cone attacks that originate from  
outside the bunker.
 

The bunker is open topped and can be entered and exited at will. If attacked the bunker has  
evasion 5 and 40 HP. It cannot be moved or deactivated once deployed.
 

                                                                                                                  


Leviathan Heavy Assault Cannon  

The Leviathan AC is a massive rotary autocannon, an enclosed multi-barrel automatic weapon fed by an  

external reservoir, usually dorsally mounted on the chassis carrying it. At its current chambering, the  
Leviathan should only be fired on automatic when absolutely necessary; IPS-N is currently working on a  
solution to meet the cannon’s needs remotely. IPS-N recommends outfitting willing squadmates with extra  

reservoirs, should their chassis have room to support it.   

Superheavy Cannon
 
2 heat (self)
 
Range 8
 
1d6 kinetic damage
 

Unlike other superheavies, this weapon can be fired as part of a skirmish action with its listed  
profile.
 
As a quick action, you can spin up this weapon’s barrels. While this weapon’s barrels are  
spinning, your mech is Slowed, but this weapon’s damage increases to 4d6+4 kinetic, it must be  
fired with a barrage action like a regular superheavy, and it gains Reliable 4. You can stop the  
spin-up as a free action at the start of your turn, but lose the increased damage until you spin the  
weapon up again.
 

                                                                                                                     
\subsection{IPS-N Lancaster}

                                         IPS-N LANCASTER  

The IPS-N LANCASTER is a mil-spec variant of an older IPS-N design, modernized and streamlined for  
military/operator use. The LANCASTER features multiple redundant systems and object/environment- 

interact projectors to facilitate pinpoint accuracy when engaging with delicate systems, damaged or intact.  
Commonly piloted by sapper and engineer-designate pilots in frontline support/specialist roles. 
 

                                                   License:
 
I. Restock Drone, Cable Winch System
 
II. LANCASTER FRAME, MULE harness, Sealant Spray
 
III. Plasma Cutter, Aceso Swarm
 

                                                LANCASTER 

  HP: 6          Evasion: 8                            Speed: 6           Heat Cap: 7        Sensors: 10 

  Armor: 1       E-Defense: 8                          Size: 2            Repair Cap: 10     Tech Attack:  
                                                                                             +0 

                                                   TRAITS: 

  Redundant Systems: Other friendly mechs of the Lancaster’s choice that are adjacent to it can spend  
  the Lancaster’s repairs as if they were their own
 
  Combat Repair: The Lancaster can spend a full action and 4 repairs in combat to repair a destroyed  
  mech, returning it to 1 structure and 1 HP 

                                             SYSTEM POINTS: 8 

                                                   MOUNTS: 

  Main/Aux Mount 

                                                CORE system 

                                                                                                            


                                                      Latch Drone  

  Known colloquially as a ‘Wingman’ drone, latch drones are companion drones carried upon and  
  deployed from a chassis. Pilots are advised against developing attachments to these drones, given their  
  high casualty rate.
 

   Integrated Mount:
 
  Latch Drone  
  Auxiliary Launcher
 
   Range 8
 
   Make a grit roll vs evasion 8 and target any friendly mech in range (still take cover and line of sight into  
   account). On hit, your target can spend up to 1 repair to heal.
 

  Active (requires 1 Core Power):
 
   Supercharger
 
   Quick action
 
  You fire your drone at a friendly mech in range, where it clamps onto the target. For the rest of this  
   scene, you take 1 heat at the start of your turn, but the targeted mech gains +1 Accuracy on all attacks  
   and checks, and is immune to the impaired, jammed, Slowed, and immobilized conditions. This effect  
   ends if you or the targeted mech is stunned or shut down. While this system is active, you cannot fire  
  your drone as a weapon (using the passive of this system). 

Restock Drone   

A simple, reliable, and sturdy drone mounting a printer, a restock drone allows for limited logistic capability  
through autosalvage: the bulk of the drone is RawMat, a generalized mix of silicates and metallic materials  
meant to be processed for high-yield printing. Pilots often call restock drones a “mech snack”.    

2 SP, Limited (2)  

Drone  
As a quick action, you can set this drone down in any adjacent space. After your turn ends, the  
drone primes. Any allied mech that moves adjacent to the drone can activate it as an interaction.  
That mech can then cool 1d6 heat, reload all weapons with the loading tag, and end one  
condition affecting it. The drone is then consumed, deactivating and disintegrating.
 

Cable Winch System   

A winch system consists of a spool of nanocarbon-weave cable mounted externally, and recovery  
subroutine software uploaded onto the recovery mech’s datamind.   

1 SP  

Quick action  
As a quick action, you can attach the cables to an adjacent mech. If the mech is shut down,  
stunned, or a willing target, this action is automatically successful, otherwise it can make a hull  
check to resist this effect. Once attached, your mech and the attached mech cannot move more  
than 5 range away from each other. One mech can tow the other, but is Slowed while doing so,  
and must successfully pass a hull check to do so. Any mech can make a successful melee or  
improvised attack to remove the cables (removed on a hit, the cables have evasion 10). The  

                                                                                                                      


cables can also be attached to the environment or any object. They are 10 length when used this  
way and can take a combined size of 6 in strain if using them to climb, etc, before they break.
 

MULE harness  
The Multiple User, Light Entanglement harness is a mass-produced version of a common battlefield  
modification that allows friendly soldiers to ride along on friendly chassis. Some systems are large and  
sturdy enough to allow for smaller chassis to accompany larger chassis; these are typically employed in  
High Altitude, Low Orbit insertions to reduce radar signatures.  

2 SP, Unique  
Your mech has mounts, straps, and hard points built to carry a total number of actors whose   
total size is less than your own (so size 1 = 1 size 1 actor, two size 1/2 actors). Actors of your  
choice that are adjacent to you can spend a quick action to mount your mech. While mounting  
your mech, they occupy your mechs’s space, move when you move and benefit from light cover.  
Any area of effect attacks that target your space will also target your rider. If your mech is  
knocked prone or is destroyed, they fall off into an adjacent space. They can dismount by  
moving normally off your mech any time.  

Sealant Spray  

2 SP  
Quick Action  

This system can be used on any actor or free space within range 5 and line of sight. It has  
different effects depending on what it is used on
 
         	Hostile actor: Make a ranged attack vs the target. On hit, the target is Slowed until the  
         start of your next turn but immediately ends any Burn affecting them.
 
         	Empty space: This creates a blast 2 area around the targeted space. The area becomes  
         difficult terrain for the rest of this scene and this puts out any fires in the area.
 
         Allied actor: 	  Your target is Slowed until the start of your next turn but can immediately  
         end any Burn effecting them.
 

Plasma Cutter
 

Plasma cutters were tools first, simple blades built to toggle and sustain a plasma sheath to make cutting  
metal easier for its user. Repeated ad-hoc use of cutters as a personal defense weapon to repel pirate  

boarding actions convinced IPS-N of the need for a mil-spec variant of the civilian tool. They developed the  
Cutter, now in its second generation. The Cutter MkII is hard-lined into the mech’s power core, with a port  
to attach power packs in case of cord severance. The cutting edge can be shortened to a knife variant, but  

is most popular in its “cutlass” option, a middling length variant that allows for a balance of reach and  

maneuverability in close quarters.    
 

Auxiliary Melee
 
1 heat (self)
 
Threat 1
 
1 energy damage +1 heat + Burn 1
 
Against objects and the environment, the cutter deals 10 AP energy damage
 

                                                                                                                     


Aceso Swarm  

The IPS-N Aceso Swarm system is a useful triage measure to address scoring and minor mechanical  
damage that results from combat engagements or negative environmental interaction. Due its low  

processor demand, an Aceso Swarm can be controlled by even a comp/con unit; this allows the pilot to  
concentrate on more complex repairs or immediate threat neutralization.    

3 SP, Unique  
Drone, Quick action  
Once per round, as a quick action at any point during your turn, your mech takes 1d6 heat and  
one other mech of your choice in your sensor range can spend 1 repair to heal.   

                                                                                                                      
\subsection{IPS-N Nelson}

                                              IPS-N NELSON  

The IPS-N NELSON brings the close-quarters doctrine espoused by ISP-N to its most pure form. The  
NELSON is built to brawl in environments too volatile for firearms or when ordnance has been exhausted.  

With its functional size, the NELSON can attack fast while remaining a difficult target to track. Layers of  
fractal-fold BULWARK plating allows for ceramic-analogous carbon flaking, effectively nulling the impact of  
incoming solid-state fire by dispersing kinetic energy across a rounded hull. This null-k plating protects the  

pilot from impact trauma, allowing for sustained combat efficacy in high-trade scenarios.   

The NELSON is an iconic IPS-N chassis, known across the galaxy as the FRAME of choice for the  

Albatross, the Cosmopolitan interstellar anti-piracy agency. Their distinctive white, gold, and red livery and  
mastery of the war pike -- as well as seeming agelessness due to time dilation -- has won both the  
Albatross and the NELSON a venerated place in Diasporan lore -- and secured the Albatross an  

endorsement contract with IPS-N in perpetuity.   

                                                     License:
 
I. War Pike, Bulwark Mods
 
II. NELSON FRAME, Thermal Charge, Armor Lock System
 
III. Power knuckles, RAMJET
 

                                                    NELSON 

  HP: 8           Evasion: 10                            Speed: 5            Heat Cap: 6        Sensors: 10 

  Armor: 0        E-Defense: 8                           Size: 1             Repair Cap: 5      Tech Attack:  
                                                                                                +0 

                                                     TRAITS: 

  Momentum: After making the boost action, the next melee attack from the Nelson deals +1d6 bonus  
  damage on hit
 
  Skirmisher: After making any attack, the Nelson can move 1 in any direction. This movement doesn’t  
  provoke reactions, ignores engagement, and doesn’t count against its movement for the turn. It can’t  
  make this move if immobilized or slowed. 

                                               SYSTEM POINTS: 6 

                                                    MOUNTS: 

  Flex Mount                          Main/Aux Mount 

                                                  CORE system 

                                                                                                                


                                             Perpetual Momentum Drive 

  IPS-N’s PMD exploits fighter-tier nearlight spooling to conserve and sustain a passive .000001LS 
  charge, able to be dumped into extant boost systems at the pilot’s command. The chassis fielding this 
  system must be heavily adapted through strengthening joints, limbs, and installing a k-comp crash 
  couch to protect the pilot from sudden g force and shear. 

  Active (requires 1 Core Power): Spool up PMD 
   Protocol 
   Once activated, this system remains active until the rest of the current scene. While its active, the free 
   movement from the Nelson’s Skirmisher trait increases to 4. 

War Pike  

A War Pike is a simple weapon. A long haft, topped with a dense, slim point, meant to puncture armor.  
Derivative of a mining pylon, the modern war pike is a sturdy, balanced, and reliable weapon, perfect for a  
charge.    

Main Melee
 
Thrown 5, Threat 3, Knockback 1
 
1d6 kinetic damage
 

Bulwark Mods  

A mark of pride for IPS-N, all proprietary mech cores feature IPS-N’s QuickMod system, a modular, legacy- 
compatible system of joints, hardpoints, and internal slots that make installing upgrades simple.   

1 SP  
Your mech has extended or armored arms or legs, redundant motor systems, or is otherwise  
reinforced for harsh terrain. Your mech ignores difficult terrain.
 

Armor Lock System  

IPS-N’s Armor Lock System is a total-body modification for a mech core that provides additional chassis  
stability when pilots are faced with a situation that puts their core under greater-than-anticipated stress.   

1 SP, Unique  

2 heat (self)  
When you take the Brace reaction, you can activate this system. Until the end of your following  
turn, enemy attacks targeting you are made with 1 additional Difficulty, you can’t fail agility or hull  
checks, be knocked back, grappled, knocked prone, or moved by any external force smaller  
than size 5. You end any grapples currently affecting you.
 

Thermal Charge  

Pilots have long made this popular modification to their pikes. Now, IPS-N is offering these pilots’  
modifications as a licensed and quality-tested suite for pan-galactic printing. A pike modified with a charge  

-- often colloquially called a “Fire Pike”-- is a simple plasma projector integrated into war pike, tuned to  
project a plasma sheath over the pike’s head.  

                                                                                                                      


2 SP  
Mod  

Limited (3)  
This mod can only be applied to a melee weapon. As a free action when you hit with any melee  
attack, you may spend a charge of this system to activate the shaped charge on your weapon  
and deal +1d6 bonus explosive damage.
 

Power Knuckles  

A simple weapon system, IPS-N’s power knuckles are a popular modification for pilots of CQB mech cores.  
Whether as shaped studs, hyperdense knuckles, or a series of magnetically-accelerated micro-rams, power  

knuckles amplify the already incredible hitting power of a mech core.   

Auxiliary Melee
 
Threat 1
 
1d3+1 explosive damage
 
On a Critical Hit, your target must pass a hull check or be knocked prone
 

RAMJET  
Air. Air and momentum. There’s a threshold that veteran Nelson pilots know well, the Point of Endless  
Momentum. When you get moving fast enough, in the right atmosphere, the air itself feeds into auxiliary  
ports on the chassis, compressing, howling out like a demon’s angry scream. The Point of Endless  
Momentum is a giant’s hand on your chest and a god’s chariot under your feet and you feel like you can  
outrun light itself and there’s nothing else like it.  

3 SP, Unique
 
Protocol
 
2 heat (self)
 
Until the start of your next turn, your mech gains +2 speed when boosting and its melee attacks  
(including rams, grapples, etc) gain knock back +2. However, your mech must move its  
maximum speed each time it moves and can only move in straight lines (it can stop if it would  
collide with an obstacle or enemy, and it can change direction between movements).
 

                                                                                                                    

\subsection{IPS-N Raleigh}
                                                             
                                              IPS-N RALEIGH  

The IPS-N RALEIGH, more so than any other mech in IPS-N’s core line, is meant to meet any enemy, any  

where, in any combat scenario. The RALEIGH is an all-rounder build that trends towards the midrange. It is  
commonly outfitted with an auxiliary hand cannon for ranged capability, a massive hammer to deal with  
anything that gets close, and the iconic, chest-mounted MJOLNIR cannon.  

                                                      License:
 
I. Hand Cannon, Breaching Charges
 
II. RALEIGH FRAME, ROLAND Chamber, Bolt Thrower
 
III. UNCLE class NHP, Kinetic Hammer
 

                                                                                                                  


                                                    RALEIGH 

  HP: 10          Evasion: 8                              Speed: 4            Heat Cap: 5         Sensors: 10 

  Armor: 1        E-Defense: 8                            Size: 1             Repair Cap: 5       Tech Attack:  
                                                                                                  +0 

                                                      TRAITS: 

  Full Metal Jacket: If the Raleigh makes no attack rolls during its turn, it can re-load all weapons with  
  the loading tag at the end of its turn as a free action.
 
  Shielded Magazines: The Raleigh can still make ranged attacks if it is Jammed. 

                                               SYSTEM POINTS: 5 

                                                     MOUNTS: 

  Aux/Aux                             Flex Mount                              Heavy Mount 

                                                  FRAME system 

                                           IPS-N M35 ‘Mjolnir’ cannon  

  IPS-N’s M35 MJOLNIR cannon is a carryover from Northstar’s WATCHMAN line of defensive weapons,  
  reworked for frontline combat. The MJOLNIR is a hard-mount, multi-barrel auxiliary cannon that uses  
  magnetic acceleration to fire stacks of airburst projectiles at its target. It is an impulse weapon, a system  
  tied to a pilot’s second-tier neural processes as dictated and coached by their partner Comp/Con or  
  NHP; even in death, a pilot’s MJOLNIR will continue to identify and attack hostile targets until total  
  systemic failure. For this reason, the MJOLNIR is often referred to as a deadgun, one of many such  
  weapons common among CQB-oriented pilots.        

  Integrated Mount:
 
  M35
 
  Main CQB
 
  Range 5, Threat 3
 
  4 kinetic damage
 

  Active (Requires 1 Core Power):
 
  Thunder God
 
  Protocol
 
  Until the end of the current challenge, if you didn’t fire your M35 on your turn, it gains 2 more rounds in  
  the chamber at the end your turn (you can use a d6 to track this). It starts with 0 rounds in the chamber.  
  When you next fire the weapon, it fires all chambers, for 4 damage per chamber. The M35 has six  
  chambers, for a maximum of 24 damage. If 4 or more chambers are fired at once, this weapon gains  
  the AP tag and any target struck must pass a hull check or be knocked prone. 

Hand Cannon  

                                                                                                                 


The IPS-N HAND CANNON is a licensed version of GMS’s Pattern I Pistol, chambered for a heavier caliber  
of round. This modification requires a change from the belt-fed system of the P1P to a magazine-based  
system, limiting the number of rounds that a mech can load at a time.   

Auxiliary CQB
 
Loading, Reliable 1
 
Range 5, Threat 3
 
1d6 damage
 

Breaching Charge  
A breach/blast charge is simply a shaped, milspec pattern of IPS-N’s generalist/civilian blasting charge,  

meant to crack asteroids. The IPS-N BB features a far more pure blend of high explosives designed to  
cause massive traumatic damage to mechs and other hardened structures.   

2 SP, Limited (2)  
Grenade, Mine  
If thrown as a quick action, the charge explodes on impact. If planted as a mine, it can be  
detonated as another quick action by whoever planted it (or detonates normally when a target  
moves adjacent). The charge deals 2d6 Energy damage to targets in a burst 1 area around the  
charge. Targets can pass an agility check to reduce this damage by half. Against objects, this  
charge does 10 AP kinetic damage.  

ROLAND Chamber  
Packed into sealed, self-contained cylinders, IPS-N’s ROLAND rounds are heavy shells purpose-built for  
any kinetic weapon that can accept cylindrical magazines. Packed with a non-O2 dependent accelerant,  
ROLAND Chambers can be used to reliably send air-or-impact burst shells downrange.   

For use in outdoor or Certain-Kill environments; use extreme caution when firing in pressurized spaces.  

2 SP, Unique  
When you reload, your very next attack deals +1d6 bonus damage as explosive damage and  
targets affected by this bonus damage must pass a hull check with 1 difficulty or be knocked  
prone.
 

Bolt Thrower  
IPS-N’s bolt thrower is a milspec variant of a civilian mining tool. A bolt thrower fires self-propelled  

explosive bolts that can be triggered manually, on a timer, on impact, on designated-depth penetration, on  
proximity, on on some combination of any allowable parameter.   

Heavy Rifle
 
Loading, Range 8
 
Reliable 2
 
2d6 kinetic + 1d6 explosive damage
 

UNCLE-class NHP  

                                                                                                                  


IPS-N’s UNCLE NHP is the result of the DARKSTAR Program, an NHP think tank funded by IPS-N’s  
Administrator Partnership. UNCLE is a pocket-AI, meant to be bound to a weapon system and assist its  
owner in peak-efficiency operation. UNCLE AI’s are currently available only as a beta system and, as such,  

owners are expected to accept all pushed updates; IPS-N waives culpability for any sub-optimal  
performance of UNCLE systems not kept current via Omninet updater.   
UNCLE NHPs are (perhaps unfairly) regarded as lesser compared to their compatriots and their inferiority  

complexes tend to display themselves as unstable personalities.  

3 SP, Unique  
Mod  
Choose 1 weapon - The weapon and its associate systems gain an NHP that has control over  
that specific weapon. This is not a full AI and can’t control your mech or become unshackled  
(and doesn’t have the AI tag).
 

It can attack by itself once on your turn as a free action, using the mech’s attack bonuses but  
with +2 Difficulty. It can’t fire a weapon that has already been fired this turn, and if you fire a  
weapon with UNCLE you cannot use it until the start of your next turn.
 

Kinetic Hammer  
A Kinetic Hammer is, in the trend of IPS-N weapons, a simple tool. A supermassive, shaped head fused to  

a long haft, the Hammer impacts with enough force to create massively traumatic pressure waves upon  
landing a successful blow.   

Heavy Melee
 
Reliable 3
 
Threat 1
 
2d6+2 kinetic damage  

                                                                                                                    


  \subsection{IPS-N Tortuga}

                                                   IPS-N TORTUGA  

The TORTUGA is IPS-N’s short-to-medium range core-line mech. Conceived, tested, and perfected in the  
void of deep trade space, the TORTUGA is made to breach and clear the spinal columns of capital ships,  
carriers, and hostile stations. The TORTUGA is built to occupy space, filling hallways with its angular bulk. It  

                                                                                                                               


defends just as effectively as it attacks, often used in a battering-ram role by boarding parties and ship/ 
stationboard marines. Conversely, the Tortuga is often employed in a defensive posture by marines seeking  
to repel boarding parties, often using its ablative brachial structures to shield troopers from incoming fire.     

                                                   License:
 
I. Automatic Shotgun, Siege Ram
 
II. TORTUGA FRAME, Throughbolt Rounds, Daisy Cutter
 
III. Pneumatic Hammer, Hyper Dense Armor
 

                                                 TORTUGA 

  HP: 10         Evasion: 6                            Speed: 3           Heat Cap: 6        Sensors: 15 

  Armor: 2       E-Defense: 10                         Size: 2            Repair Cap: 6      Tech Attack:  
                                                                                             +1 

                                                   TRAITS: 

  REFLEX: The Tortuga gets +1 Accuracy to all overwatch attacks
 
  Guardian: Allied actors adjacent to the Tortuga gain light cover  

                                             SYSTEM POINTS: 6 

                                                   MOUNTS: 

  Main Mount                        Heavy Mount 

                                                CORE system 

                                                                                                            


                                                         SENTINEL
 

  IPS-N security teams are no strangers to the danger of ship-to-ship or ship-to-station boarding actions.  
   Tight corridors, unstable gravity, dark environments, hard vacuum, and the dual threat of organic and  
  inorganic opposition forces make boarding actions some of the most deadly engagements (by  
  percentage) that one could participate in -- the winning side, according to IPS-N’s internal metrics,  
  should expect at least 30% casualties on average.  

   To lessen the cognitive burden of pilots and any NHPs or comp/cons they have installed in their chassis,  
  IPS-N developed the SENTINEL co-pilot subsentient partition. The Sentinel is a simple subsentient: a  
  flash-homunculus of an aggregate-intelligence compiled through thousands of after-action reports from  
  boarding actions, debriefings, and volunteer donors. Not a true AI, nor an NHP, the SENTINEL is a  
  robust tactical program similar to a smart weapon, though without the need to cycle it presents certain  
  tactical advantages -- namely, the ability for limited learning and best-guess predictive capabilities  
  alongside its pilot. SENTINELs are largely plain in their personalities, such that they develop, and are a  
  favorite of pilots for their no-nonsense attitude and crisp, efficient counsel.   

   The SENTINEL is currently under review by a joint USB/UDoJ-HR committee, but no formal stay on  
  production has yet been issued.    

  Active (Require 1 Core Power): Hyper-reflex mode  
   Protocol
 
   For the rest of this combat, your threat with ranged weapons increases to 5 if it was less than 5. You  
  can make one additional overwatch attack between your turns, and any target struck by your  
  overwatch attacks is immediately immobilized until the start of its next turn. 

Automatic Shotgun  

The IPS-N Deck-Sweeper Automatic Shotgun is a belt-fed scattergun, a favorite of marine pilots aboard stations and  
capital ships. It’s operation is simple and straightforward: charge, point, and fire. The single-barrel constriction allows  
for pneumatic absorption, dampening the effect of its incredible recoil, and its belt-fed action accepts many types of  
shot-and-FRAME ammunition.   
The DSAS is a mainstay among IPS-N licensed pilots.     

Main CQB
 
Inaccurate
 
Range 3, Threat 3
 
2d6 Kinetic Damage
 

Siege Ram  

The Siege Ram is another holdover from IPS-N’s pre-merger days. When Bulkheads slam closed and there  
is a need to get them open, marine pilots mount a siege ram to get the job done. Heavy, dumb, and  

unbreakable, the Siege Ram is the universal key. Carried in-hand by a qualified chassis, the IPS-N Siege  
Ram is a solid metal beam with a wedge tip, meant to be slammed into the seam of a sealed bulkhead door  

and driven home, cracking open ships and stations like a can.    

2 SP, Unique
 

                                                                                                                          


Your ram attacks deal 1d3 kinetic damage on hit. Against stationary objects, deployed cover,  
terrain, walls, or obstacles, your ram attacks instead deal 10 AP kinetic damage.
 

Throughbolt Rounds  

Throughbolt Rounds are a proprietary IPS-N invention. Throughbolts are Tungsten-jacketed, uranium core  
rounds with projection-activated plasma sheaths. When fired, the rounds ignite and project a superheated  

cone of plasma before them, creating a miniature lance effect that ensures multiple-target penetration  

through soft and hard surfaces.    

2 SP  

Mod  
Choose 1 CQB, cannon, or rifle weapon. When you fire this weapon, draw a line 3 spaces long  
from your mech, then measure its original range from the end of this line as though the attack  
was fired from that position (also measure cover and line of sight from this new position for the  
rest of the attack). This line can easily punch through walls or other barriers. Any targets hit by  
this line are also hit by the attack. The attack cannot change directions after being fired.
 

Daisy Cutter  

The Daisy Cutter is an effective, if outdated, weapon system for which many marine pilots still place print  
requisitions. The Daisy Cutter is, essentially, a massive shotgun: the pilot loads a shaped charge into the  
breech of the Cutter, drops a packed sabot down the barrel, aims, and fires a mixed hellfire cloud of  

flechette darts, bearings, and ignited magnesium strips, clearing any deck it’s been fired on.   

Heavy CQB
 
Limited (1)
 
Cone 5
 
3d6 kinetic damage. 
 
The blast cloud from firing this weapon lingers until the end of your next turn, providing light  
cover to any actor in the affected area.
 

Pneumatic Hammer  

Colloquially known as a ‘pilebunker’, built originally from blast mining equipment, the pneumatic hammer  
has been refined into a widely feared weapon - a solid-core cylinder cocked and locked in place by a  
miniaturized gravity well. When fired, the cylinder is propelled forward by a charge of superheated plasma  

through a cannon-like shaft, creating enormous kinetic force. Without proper reinforcement, the power  
created by this weapon will literally tear its wielder’s arm off.  

Main Melee
 
Loading
 
Threat 1
 
1d3+5 kinetic damage
 
On a Critical Hit with this weapon, your target must pass a hull check or be stunned until the end  
of its next turn.
 

                                                                                                                    


Hyper Dense Armor  

IPS-N HyperDense Armor is built for use in space. As the name implies, the HyperDense system is forged  
without respect to the gravitational constraints mechs may face down a gravity well; many pilots flying  

cores equipped with HyperDense armor are shocked to experience the difference in piloting their mechs  
down a well versus in the null-gravity of space.   

3 SP  
Unique, Protocol, 2 heat (self)
 
You may activate or deactivate this armor system’s activation protocols at the start of your turn.  
While active, it hardens into a shimmering, reflective surface and offers unparalleled protection,  
granting you resistance to all damage from attacks further away from range 5 of your mech.  
However, your mech is Slowed while it is active. 
 

                                                                                                                      

\subsection{IPS-N Vlad}
                                                                 
                                                     IPS-N VLAD  

The IPS-N VLAD is a variant of the IPS-N NELSON, built to handle hardened targets that would present  

strategic difficulty for the NELSON platform. The VLAD features a suite of myth-inspired weaponry and  
heavy armor and is meant to take a frontline role, absorbing fire from dangerous targets in order to protect  
its allies while lining up the perfect shot.   

                                                                                                                          


                                                      License:
 
I. Snare Trap, Impact Lance
 
II. VLAD FRAME, Nail Gun, Caltrop Launcher
 
III. Combat Drill, Charged Stake
 

                                                       VLAD 

  HP: 10          Evasion: 8                              Speed: 4            Heat Cap: 4         Sensors: 5 

  Armor: 2        E-Defense: 8                            Size: 1             Repair Cap: 4       Tech Attack:  
                                                                                                  +0 

                                                      TRAITS: 

  Dismemberment: When the Vlad successfully immobilizes a target, that target is also Shredded for the  
  same duration
 
  Shrike Armor: When the Vlad is attacked by any actor within range 3, the attacker takes 1 AP kinetic  
  damage before they attack 

                                                SYSTEM POINTS: 5 

                                                      MOUNTS: 

  Flex Mount                          Main Mount                               Heavy Mount 

                                                   CORE system 

                                                    Shrike Armor 
  A nod to the pre-Fall namesake of the VLAD, Shrike armor plating bristles with shaped spikes, hardened 
  with chromium/tungsten alloy tips. Strategic studding places Shrike tips in high-likelihood kinetic 
  encounter areas: gauntlet covers, manipulator joint covers, shoulder plating, and so on. Primarily a 
  defensive modification, Shrike armor is uncommon among Coreside pilots, and seen as a mark of 
  underdeveloped -- if terrifying - tactics. 

  Active (requires 1 Core Power): 
  Tormentor spines 
  Protocol 
  Until the end of the current challenge, you gain resistance to all damage from within range 3, and your 
  damage from this mech’s Shrike Armor trait increases to 3 AP kinetic damage. 

Snare Trap  
The IPS-N WEBJAW Explosively-Accelerated Filament system is a deployable all-theater perimeter  
defense system designed to arrest hostile movement in pre-determined kill-corridors. Deployable by hand  
or launch tube, the WEBJAW EAF system consists of a cluster of filament anchors scattered across an  
area. When triggered remotely or by a series of programmable physical, electronic, or chemical triggers, the  
anchors target and fire at the triggering foe, embedding hard-tip barbs deep inside both hard and soft  
targets. The barbs, anchored to their bases by arachnosilk-analog filament, immobilize and entangle the  
target.  

                                                                                                                  


1 SP  
Mine, Limited 1  

This trap triggers when any actor passes directly over it. The target must pass a hull check or  
take 2d6 AP kinetic damage and become immobilized. Once triggered, the trap becomes an  
object with 10 HP and 5 evasion, and immobilizes its target as long as it is not destroyed.
 

Impact Lance  
The Impact Lance is a milspec variant of a common mining tool: the single-use, proximal-distance chemical  
survey laser. IPS-N’s military variant mounts a series of Impact Lances on a brachial or thoracic carriages,  
leaving a chassis’ manipulators free to field other weapons and systems; the Lance can be wired directly  
into a chassis’ core, or charged with single-use chemical batteries.   

The lances fire for a microsecond, burning through their stored charge in a milisolar burst of light that stabs  
out in a tight, pulsed beam capable of searing through multiple meters of hardened bulkhead.  

Main Melee
 
Threat 2
 
1d6 energy damage
 
This weapon attacks in a line drawn between its target and your mech, attacking all other actors  
in between, but deals +1 heat to your mech for each target hit past the first
 

IPS-N “Impaler” Nailgun  

The milspec Nailgun utilizes non-combustible, sabot-jacketed two-stage macroflechettes to pierce even  

the most substantial of armor. First catapulted from its launcher, the macroflechette’s sabot disengages on  
approach to its target, triggering a second stage where internal propulsion drives the macroflechette  
forward with incredible velocity. Against soft targets, over-penetration is certain: IPS-N advises pilots  

employ this weapon platform only when the area behind the target is clear of allies and/or noncombatants.   

Main CQB
 
1 heat (self)
 
Range 8, Threat 3
 
1d6 kinetic damage
 
On a Critical Hit, the target of this attack must pass a hull check or be immobilized until the end  
of its next turn
 

Caltrop Launcher  
A wicked anti-organic, anti-vehicle, proximity denial system, chassis-mounted caltrops are fired in great  
clouds of shimmering metal (or deployed in long swathes) to blanket an area.   

IPS-N’s HX-CAL caltrop system adds small, shaped explosives to the mix of hardened pyramids.  

1 SP, Unique
 
Quick Action
 
When this system is activated, your mech targets a free space within range 5 and blankets a  
blast 1 area centered on that space with explosive caltrops. That area becomes difficult terrain,  

                                                                                                                     


and mechs moving across the space (voluntarily or otherwise) take 1 AP explosive damage for  
each space they move.
 

Combat Drill  

The IPS-N combat drill is a brutal close combat weapon, powered by a massive catalyst pack mounted  
externally on a mech core. The drill is tipped with micro-plasmatic projectors designed to pre-treat the  

target to ensure bit purchase and facilitate drill penetration.   

Superheavy Melee
 
Overcharged, AP
 
Threat 1
 
3d6 kinetic + 1d6 energy
 

Charged Stake
 

Built from gear meant originally for blast mining, this enormous, improvised system is loaded and cocked  
prior to embark into a specially primed chamber. It is designed to penetrate and immobilize hardened  

targets, then send powerful, vaporizing charges into its vulnerable internal systems.  

2 SP
 

Full Action  
This brutal system can be used against any adjacent target. That target must pass a hull check  
with 1 difficulty or take 2d6 energy damage damage and become immobilized and impaled. At  
the end of each of its turns, the target can repeat this check to end the effect on itself, otherwise  
it takes 3 AP energy damage and remains immobilized until it makes the check successfully. Only  
one target can be immobilized by this system at once, but it can be picked up as a quick action.
 

                                                                                                                    

\section{Basic Quick Actions}

                       BASIC QUICK ACTIONS  
\subsection{Skirmish}

                                             SKIRMISH  

When you take the skirmish action, you attack with a single weapon from your mech.   
        - You can also make an attack with another auxiliary weapon from the same mount. That  
        weapon can’t deal bonus damage. Auxiliary weapons are light and can be used to make  
        quick, numerous attacks.
 
        	- Superheavy weapons are too cumbersome to be fired with a skirmish action and must  
        be fired as part of a barrage action.
 
\subsection{Boost}

                                                BOOST  

When you take the boost action, you can move your speed. Boosting allows you to move again,  
in addition to taking a move action on the same turn. Certain talents and systems only activate  
on boosts (not regular movement).
 
\subsection{Ram}

                                                 RAM  
Ramming is a melee attack made against an adjacent target with the aim of knocking down or  
back an enemy mech.   
If your attack is successful, your target is knocked Prone and you may also knock your target  
back up to 1 space directly away from you.  
\subsection{Grapple}

                                              GRAPPLE  
When you Grapple, you attempt to grab hold of an enemy mech and overpower it, disarming,  
subduing, or damaging it so that it cannot do the same to you.   

In order to perform a Grapple, choose an adjacent target and make a melee attack. On hit:  
    -   Both parties are engaged  
    -   While grappled or grappling, neither party can boost or take reactions  
    -   The smaller party is immobilized, but moves when the larger party moves, mirroring their  
        movement. If both parties are the same size, they can make a contested hull check when  
        they attempt to move, counting as the large party for their turn if they win.  
    -   The grapple breaks if either target breaks adjacency (is knocked back for example)  
    -   The attacker can end the grapple as a Free Action, and the defender can end the grapple  
        as a quick action by making a successful hull or agility check.  

                                                                                                        


    -    If there are multiple parties involved in a grapple, the same rules apply, but when counting  
         size, count up all opponents of a side in a grapple. For example, if my all and I are both  
         size 1 and grappling a size 2 target together, we would count our total size (2) and could  
         attempt to drag our target around.   
\subsection{Quick Tech}

                                              QUICK TECH  

The Quick Tech actions cover electronic warfare, countermeasures, and other actions that can  
be taken by a pilot, often aided by their mech’s powerful computing and simulation cores. Many  
pilots choose NHP (non-human person) assistants or more conventional comp/con units to help  
them with these tasks. All mechs have access to the basic tech actions. Further tech actions can  
be enhanced by taking systems that upgrade them. 
 

Some tech actions are attacks (often called tech attacks) and benefit from generic bonuses to  
attack rolls. All tech actions must choose a target within Sensor Range to be effective, and roll  
systems vs. e-defense. To use a tech action, choose a target in your sensor range (including  
yourself) and choose one of the following options:
 

Bolster  
You use the formidable core processing power of your mech’s systems to boost one other  
target’s systems. That target can take +2 Accuracy on its next skill check of any kind before the  
end of its next turn. A mech can only benefit from bolster once at a time.
 

Scan
 
You can use your mech’s powerful internal systems to deep scan your enemies.  
To Scan, make a tech attack against a target in your sensor range. On a successful attack, ask  
your GM to reveal one of the two to you:
 
             -   Your target’s full statistics (HP, Speed, Evasion, Armor, HASE, etc), weapons, and  
                 systems
 
             -   Hidden information about the target, such as information caches it is carrying,  
                 current mission, pilot ID, etc. 
 
This information is only current when you receive it (for example, if the target loses HP again,  
your information won’t update).
 

Lock On  
Make a tech attack against a target in range. On hit, the target suffers from the Lock On  
condition, enabling some systems and talents. Any attacker can end Lock On on a target when  
they attack that target to gain +1 Accuracy on their very next attack roll against that target.
 

Invade
 
Make a tech attack against a target in range. On success, your target takes 1d3 heat and you  
may choose one of the following options:
 

                                                                                                                


         Fragment Signal/Feed Misinformation: You feed false information, obscene messages,  
         or phantom signals to your target’s core computer, inflicting the Impaired Condition on  
         your target until the end of their next turn.
 

         Aggressive Code: You attack your target’s servos and engines, inflicting the Slowed  
         condition on your target until the end of their next turn.
 

         Attack systems: You go for the throat, the core computer. Inflict an additional 1d3 heat  
         on your target
 
\subsection{Hide}

                                                      HIDE  

In order to perform the Hide action, you need cover or concealment. The cover needs to be large  
enough to totally conceal your mech (such as a smoke cloud or building) or you won’t be able to  
hide. Lack of line of sight is always sufficient, and if you’re invisible, you can always attempt to  
hide.
 

Hiding is always successful. After you hide, you gain the hidden condition. A hidden target can’t  
be directly targeted by attacks or hostile actions, but can still be incidentally hit by attacks that  
target an area. NPCs cannot perfectly locate a hidden target but only know their approximate  
location. 
 

Performing any attack (melee, ranged, or tech), the boost action, or taking a reaction will break  
hiding. You can take other actions as normal. You must end your turn in cover to keep hidden.  
You automatically lose hidden if you end your turn in a place where you wouldn’t benefit from  
cover (ie, a mech comes around a wall and can now draw unbroken line of sight to you), your  
cover is destroyed, or you move from cover. If you’re hiding and invisible, you also lose hidden if  
you lose invisibility.
 
\subsection{Search}
                                                   SEARCH  

To detect a hidden target takes a quick action and makes a contested check.  
         Mech: The searching party needs you to be in their sensor range and makes a systems  
         check. A hidden mech makes an agility check.
 
         Pilot: The searching party needs you to be in range 5 and makes a pilot skill check, using  
         skills such as notice. A hidden pilot makes a skill check and can use bonuses such as  
         infiltrate.
 
Once a hidden target is detected, it loses the hidden condition.
 
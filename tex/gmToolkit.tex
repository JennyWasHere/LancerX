\section{GM Toolkit}
                                     GM TOOLKIT  

Here are some tools for running your game and creating different and more interesting scenarios  
for your players to go through. These rules will help clarify certain situations, customize or add  
additional detail or flavor to your game.
 
\subsection{Piloting a mech}
                                        PILOTING A MECH  
Piloting a mech is a simple matter of mounting it as an action. If a mech is powered off (shut  
down), it must also be booted as an action.
 

If you pilot a mech you are not licensed for (such as an enemy mech) the lack of correct  
neurological interfacing means that mech is permanently impaired and Slowed while you pilot it.  
Pilots piloting an NPC mech only have access to NPC actions, not those of a player character (ie,  
they cannot take the Stabilize action, overcharge, etc).
 
\subsection{Objects and Damage}
                                    OBJECTS AND DAMAGE  

All objects, cover, deployables, etc, have evasion 5 and HP equal to 10x their total size (so a size  
4 object or an object made up of 4 size 1 objects would have 40 HP). Some deployable objects  
(such as drones) specify different armor or HP, which supersedes this rule. This could cover  
terrain, walls, or any other construction on the battlefield. If the object is tough or hardy (such as  
solid rock), you could give it 1 or 2 armor. If it’s fortified, such as a bulkhead, bunker, or starship  
freighter wall, give it 3 armor.  

You can ignore this rule at your leisure when it applies to entities not created by players or out of  
combat. If players want to bust through a wall to get the drop on their enemies, you can always let  
them make a hull check to do it without firing a shot.  

\subsection{Changing Core Assumptions}
                             CHANGING CORE ASSUMPTIONS  

The mechanics of LANCER assume a couple of things that might not be present in your  
campaign. If you want to tweak these things it’s entirely up to you. The following tools can help  
you change around some core conceits of the game.
 

                           PILOTS DON’T HAVE ACCESS TO A PRINTER
 

It’s assumed that pilots generally have access to a printer to create new mechs. This might not  
be the case in your campaign or even your setting, however! Maybe players are outlaws or  
renegades, with limited access to resources. Maybe the printer in their area is broken or  
damaged. Maybe they are operating on the fringes of civilization, where any kind of technology is  
hard to come by.
 

                                                                                                          


Printing a mech allows a player to get back in the game, so to speak, when their mech is  
destroyed. Remember that players can repair and rebuild their mech completely, as long as it is  
at least partly intact and they have access to it, whenever they take a full repair, regardless of  
whether they have a printer or not. 
 

If pilots don’t have a printer and their mech is destroyed (or they can’t access the mech), use the  
Power at a Cost tool at the beginning of this section (the goal: I want to rebuild my mech) to get  
access to people, materials, a workshop, etc where mechs can be manually built or repaired.  
Building a mech can also be a downtime activity (see the section above).
 

                                          DEATH IS MORE LIKELY
 

Here’s an optional rule you can use if you want to slightly tweak LANCER’s default ‘heroic death’  
rules: 
 
             -    If you take more than your maximum HP in a single hit (after armor) as a pilot,  
                 you’re dead, no matter what. 
 
             -    If you’re dead, that’s it. No cloning or revivification.
 
             -    If you take two points of structure in a single hit, your mech is destroyed, no  
                  matter what. 
 
Make sure you know what kind of game your players and you are playing before adding this sort  
of rule in.
 

                                    I WANT TO SIMULATE CURRENCY  

LANCER does away with currency management like in other RPGs in favor of tying everything to  
the leveling system. If your players want to buy something, they can just buy it (unless it’s  
expensive or rare, then do some role playing or use Power at a Cost). It’s assumed pilots are  
still paid (in manna, currency etc), you just don’t track it.  

If you don’t like that system, want something more granular, or want something to replace the  
License Level system, you can track Manna instead. Maybe your pilots don’t have benefactors or  
access to a market where they can freely buy mech licenses, for example.
 

                                                     MANNA  

Manna is a universal currency in the canon of LANCER promoted by Union to integrate client  
states and regulate business, in common use in certain parts of the galaxy.
 

Manna is represented by a capital M preceding the denomination, like so: M1, M2, M3, M100,  
M500, and so on. Manna is a digital currency, though it has been localized in some areas as a  
physical currency. There are also fractions of M1: M.75, M.50, M.25, M.10, and M.05.
 

Here’s what certain things typically cost at average purchasing power in Manna:
 

                                                                                                                


M.01: A cup of black coffee. Beans were grown in zero-g so it doesn’t taste the best.
 
M.05: A beer. Probably artificial but the spacers like it that way.
 
M.10: A decent, hot meal.
 
M.25: A night’s stay in a station capsule, pretty damn cramped and noisy
 
M.50: Standard bribe to gatesec
 
M1: Ticket into an exclusive offworlder nightclub
 
M2.5: Assault rifle, lightly used, sights are slightly crooked
 
M10: Personal kinetic shielding, generally reliable
 
M100: A military grade hard suit
 
M1000: A one-seater starship
 
M10,000: A full starship, crew of 5
 
M1,000,000: A freighter or warship
 
Anything higher than M1000 is usually difficult to get your hands on.
 

                                       USING MANNA TO LEVEL  

If you want to set a cost on mech parts or licenses, you can set a manna cost instead of using  
license level for certain licenses. Doing so effectively changes the leveling system to be based on  
manna, so keep that in mind.
 

To rent (use) a piece of equipment from a license for one mission costs M100 for rank I, 300  
for rank II, and 900 for rank III.   

To buy a piece of equipment costs M250 for rank I, 500 for rank II, and 1000 for rank III. 
 

If you rent a piece of equipment, it’s gone after one mission. If you buy a piece of equipment, it’s  
not re-printed if your mech is destroyed. Renting or buying a weapon doesn’t level up a player  
(they don’t get the FRAME unlocks). You can’t rent or buy a mech FRAME, you only get them by  
permanently unlocking them (as if you’d leveled up normally).
 

To permanently unlock a rank of a license, it costs 1500 (no matter the rank). If you  
permanently unlock a license, you level up (using the same leveling rules, getting 1 core point, 1  
talent point, and possible targeting bonuses, system points, or core mounts). You get access to  
all the gear from that license permanently. You can re-print anything you’ve permanently  
unlocked. Permanently unlocking a license is the same as ‘buying’ a mech so if players want to  
‘buy’ a mech FRAME, tell them it’s going to cost about 1500 to get the rank I license to access it.
 

Manna rewards could vary per mission, but if you want to keep the same leveling pace, you  
should award players about 1500 mana per mission (with more or less at your discretion).  

\subsection{Engagement}
                                         ENGAGEMENT
 

If you want to mix things up in a mission where the starting situation on the ground is unclear (a  
hot drop, an invasion, a foray into enemy territory), you can use the engagement rule.
 

                                                                                                         


Engagement happens right before the Boots on the Ground step of mission. Make an  
engagement roll where everyone can see. This is a simple d6 roll. Roll it and consult the  
following chart to establish what the situation is like the moment players get there.
 

 D6                                                    Starting Situation 

 6                                                     Situation normal, no complications other than  
                                                       expected 
 4-5                                                   Minor complications or unwelcome surprises 

 2-3                                                   Major complications or unwelcome surprises 

 1                                                     Situation FUBAR 

Engagement cuts out unnecessary planning or stalling and cuts right to when the players arrive  
on the scene. When we make an engagement roll, we immediately establish a situation and put  
the players in that situation, ready to take action and respond. 
 

This doesn’t have to throw the players right into combat (and probably shouldn’t the majority of  
the time). For an example, let’s say the players have embarked on a mission to escort a refugee  
caravan through a heavily guarded checkpoint manned by local partisans. The GM decides the  
moment players get boots on the ground is when they meet up with the caravan outside of the  
checkpoint. Based on the engagement roll, it could go the following ways.
 

6 - No major issues, the caravan is unmolested and ready to move
 
4-5 - The caravan is far larger than the players initially expected. It will move slowly and become  
hard to guard.
 
2-3 - The caravan is delayed and the players will have to track it down or wait under threat of  
bandit attack
 
1 - The caravan is under direct bandit attack the moment players arrive on the scene.
 

                                    Changing the engagement roll
 

The engagement roll can be increased by adding extra d6s (and choose the highest). It can also  
be decreased by subtracting dice. If the total pool is 0 or lower, roll two d6s and choose the  
lowest.
 

Check the chart below for ideas on how to modify the roll. This is mostly qualitative, based on  
the nature of the mission, but if the roll’s going to be adjusted, it should be fairly obvious for both  
the players and the GM (it can’t be arbitrarily changed).
 

Most of the time engagement should just be a straight roll (1d6).  

                                         Engagement modifiers
 

                                                                                                          


 Situation                                                                             Effect 

 The mission is in an exceptionally safe or stable location                            +1d6 

 The mission is in a notably unstable, dangerous, or distant location                  -1d6 

 The characters have good scouting, information, or details about the                  +1d6 
 mission 
 The characters have exceptionally poor information about the mission                  -1d6 

 Powerful forces are contesting or helping the players on their mission                -1d6/+1d6 

 The mission is routine                                                                +1d6 

 The mission is an emergency, impromptu, or rushed                                     -1d6 

\subsection{Faction Tracker}
                                       FACTION TRACKER  
There are many factions in the world of LANCER, many of which are outlined in the official canon  
below. You may, during the course of your game, find it relevant to keep track of factions in your  
game (or may run entire games revolving around factions). If you want to codify things a bit, you  
can use this tool.  

Factions can be tracked simply by Power and Hold. Power is the wealth, force, and influence of  
the faction, simply put. Hold is how resilient that faction is (strong, normal, or weak), how well it  
holds on to that position.  

Power has the following ranks (it’s not linear), as well as some examples.  

            Power      Scale             Examples 

            -3         Sub-local        A gang, small militia, or militant group 

            -1         Local            A small colony, a huge bandit gang, a small military 

            0          State            A ruler, warlord, or king; a pirate lord, a mercenary company,  
                                        a large colony, a reaver pilgrimage 

            1          Planetary        A unified planetary government, a god-king, a small  
                                         planetary-state, a pirate haven 

            3          System           A major shipping company, a trade collective, a minor corpo- 
                                        state 

            5          Multiple         A major corpo-state such as Harrison Armory, a Karrakin  
                       Systems          trade barony, Aun Ecumene, a pre-collapse civilization 

            10         Galaxy            Union 

            15         Metaphysical      RA 

The players are probably a faction with power -3 to -1.  

                                                                                                          


If a faction undertakes a major project that does not bring them into direct conflict with another  
faction, such as exploration, mining, trade, construction, research, expansion, etc roll 2d6+ that  
faction’s power to see how it goes. On double 1s, the action fails no matter what. On a total result  
of 2-6, the action is still in progress, or a failure. On a 7-9 the action is successful, but might take  
more time or resources than expected. On a 10+, the action is flatly successful.  
If two factions perform an action that would bring them into conflict with each other (such  
as a war), each rolls 2d6 and adds their power. On double 1s, the faction loses no matter what,  
otherwise the faction with the higher result scores victory (however that is defined). This doesn’t  
have to be direct conflict but could be a trade war, bidding contest, bid for influence, race for  
resources, etc.  

If a faction has strong hold, they roll 3d6 and pick the highest. If a faction has weak hold, they roll  
3d6 and pick the lowest. If a faction has normal hold, they roll as normal. Hold depends on how  
well-entrenched a faction is. A government such as Union or a planetary government usually has  
strong hold. A collapsing state, chaotic bandit gang, or unorganized military fleet has weak hold.  
Any other faction has normal hold. Any faction that goes to war immediately goes to weak hold. If  
a faction is insurgent (a rebellion, secret operation, etc), they always have strong hold, but lose  
that hold if they rise to power 1 or above.  

You can use this tool to check how well factions undertake certain actions to provide a sense of a  
living world for the players, or even allow the players to influence the outcome of events. If a  
faction has some major advantage or disadvantage that the players grant them, you could give  
them strong or weak hold, depending on the player’s actions. It might be possible for factions at  
weak hold, if they lose a conflict, to go down a rung on the power ranking (from planetary to state,  
for example).  

You might find it useful to track the faction’s attitudes towards the players as the story progresses.  

                                                                                            
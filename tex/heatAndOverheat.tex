\section{Heat and Overheating}

  HEAT \& OVERHEATING  

Heat represents the stress of combat on a mech’s electronic systems and mechanical  
components. Generally a mech is equipped with heat sinks, shunts, and coolant systems and to  
operate within factory defined standards without generating heat. However, combat and  
activated abilities can tax your mech’s heat dispersal systems to the point of causing actual  
damage. Electronic warfare attacks, environmental hazards, weaponry, and overcharging can all  
cause heat buildup.
 

Each Mech has a Heat Capacity that determines how much heat they can handle without things  
getting dangerous. It can be increased through certain systems and by improving a mech’s  
engineering score. A mech with a negative bonus to heat capacity has less than a mech with no  
bonus. A mech reactor also can take a certain amount of stress before its reactor core is  
breached and it starts to completely melt down. Most mechs have 4.
 

When a mech takes Heat, mark it off. If you gain heat that puts you up to your heat capacity or  
over, check 1 reactor stress, then make an overheating check on the OVERHEATING chart by  
rolling 1d6 per point of stress you have. If rolling multiple dice, choose the lowest result. Then  
your mech fully cools, erasing all heat from the heat gauge. You take any heat that ‘spills over’ to  
your gauge again. This could cause you to overheat more than once.
 

                                                OVERHEATING
 

 ROLL       RESULT                  EFFECT 

 5-6        EMERGENCY               Cooling systems recover and manage to contain the peaking heat  
            SHUNT                   levels. However, your mech is impaired until the end of your next turn. 

 2-4        POWER PLANT             Your mech’s power plant becomes unstable, ejecting jets of plasma.  
            DESTABILIZE             Your mech is Jammed until the end of your next turn 

 1          MELTDOWN                This result has different outcomes depending on how much reactor  
                                    stress your mech has remaining.
 
                                    3+ - Your mech is immediately shut down
 
                                    2 - Your mech must pass a engineering check or suffer a reactor  
                                    meltdown at the end of 1d6 turns after this one (rolled by the GM). You  
                                    can reverse it by taking a full action and repeating this check.
 
                                    1 - Your mech suffers a reactor meltdown at the end of your next turn 

 Two 1s     IRREVERSIBLE            Your reactor goes critical. Your mech will suffer a reactor meltdown at  
            MELTDOWN                the end of your next turn. 

                                               COOLING HEAT  

                                                                                                                


You can reset your heat gauge by taking the Stabilize action in combat or using other systems.  
You also automatically cool heat when you rest or full repair. Whenever you cool heat, your gauge  
resets, clearing all heat. 
 

                                            The DANGER ZONE  

When a mech has 1/2 of its total heat capacity filled, it’s in the danger zone. Certain mech  
weapons and talents only activate in this area. While a mech is in this zone, it’s visible - parts of  
your mech will be glowing, smoking, or steaming. Reactor vents or other cooling mechanisms  
might be visible hot or working overtime.
 

                                              CORE BREACH  

If you check your last (typically 4th) stress box, your mech enters the CORE BREACH state. In  
this state your gauge does not reset, you can no longer cool, and whenever your mech takes  
heat, it makes an overheating check. You can exit this state by resting or taking a full repair.

\subsection{Reactor Meltdown}

                                      REACTOR MELTDOWN  

Certain critical and overheating table results can cause a reactor meltdown. This can be  
immediate, or involve a countdown (in which case update the countdown at the start of the  
round. The meltdown triggers when specified). When a mech suffers a reactor meltdown, any  
pilot inside immediately dies, the mech is immediately vaporized in a catastrophic eruption (it  
becomes completely unrepairable), and any mechs inside a burst 2 area centered on the mech  
must pass an agility skill check or take 4d6 explosive damage, and half on a successful save.
 

                                                               